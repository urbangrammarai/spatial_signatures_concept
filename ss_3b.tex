\subsection{Building blocks: the Enclosed Tesselation}
\label{ssec:ss_et}

% What
This section proposes a novel and theoretically-informed delineation of space
to support the development of spatial signatures.
% Why
%% Spatial unit is a big deal
Since spatial signatures are conceptualised as highly granular in space,
considering the ideal unit of analysis at which to measure them is of utmost
importance.
%% If improper --> MAUP big time
%% If done right --> a direct way to learn about geography

% How: Requirements of a good unit
An ideal unit of analysis for spatial signatures will partition space into
consistent blocks of built \textit{and} lived environment. In other words, a
successful candidate will need to meet the following XXX characteristics:
%% Undivisable: like the atom for urban fabric
%% Internally consistent: contain only one type of fabric
%% Exhaustive: every portion of a geography should be covered and assigned into a single unit

% Existing options
%% Administrative areas (cite Taubenbock 2020)
%% Uniform grids (cite papers on imagery for squared grids and mention H3)
%% Morphometric units --> MF to fill in here

% Proposal: the ET
%% One liner of what it is
%% Two foundational blocks:
%%% Enclosures (delimitters)
%%% Building (subdelimitters) --> use morphometric lit to justify why buildings are so relevant

%% Process of delineation:
%%% Gather all enclosing features (streets, rivers, railways, etc.)
%%% Build enclosing geography
%%% subdivision it based on building presence

% [Build illustration in the description]

% Why --> meets every requirement and extends the previous state-of-the-art
