\subsection{Building blocks: the Enclosed Tesselation}
\label{ssec:ss_et}

% What
This section proposes a novel and theoretically-informed delineation of space
to support the development of spatial signatures.
% Why
%% Spatial unit is a big deal
Since spatial signatures are conceptualised as highly granular in space,
considering the ideal unit of analysis at which to measure them is of utmost
importance.
%
This process involves identifying the fundamental building block that
partitions space in a way that may later be meaningfully aggregated into
organic delineations of spatial signatures.
%
This step is worth spending enery and effort for two main reasons.
%% If improper --> MAUP big time
First, if ignored, there is an important risk of incurring the modifiable
areal unit problem (MAUP, \citealp{openshaw1981modifiable}). The urban fabric is
not a spatially smooth phenomenon; rather, it is lumpy, irregular and operates
at very small scales.
%
Choosing a spatial unit that does not closely match its distribution will
subsume interesting variation and will hide features that are at the very
heart of what we are trying to capture with spatial signatures.
%% If done right --> a direct way to learn about geography
Second, and conversely, we see adopting a meaningful unit a step of analysis
itself. Rather than selecting an imperfect but existing unit to try to measure
and characterise spatial signatures, delineating our own is an opportunity in
itself to learn about the nature of urban tissue and better understand issues
about distribution and composition within urban areas.

% How: Requirements of a good unit
Let us first focus on what is required from an ideal unit of analysis for
spatial signatures. We need a partition of space into sections of built
\textit{and} lived environment that can later be pieced together based on
their characteristics. The result will feed into an organic delineation that
captures variation in the appearance and character of urban fabric as it
unfolds over space.
%
To be more specific, a successful candidate for this task will need to fulfill
at least three features: indivisibility, internal consistency, and exhaustivity.
%% Undivisable: like the atom for urban fabric
An ideal unit will need to be \textit{indivisable} in the sense that if it
were to be broken into smaller components, none of them would be enough to
capture the notion of spatial signature.
%% Internally consistent: contain only one type of fabric
Similarly, every unit needs to be \textit{internally consistent}: one and only
one type of signature should be represented in each observation.
%% Exhaustive: every portion of a geography should be covered and assigned into a single unit
Finally, the resulting delineation needs to be geographically
\textit{exhaustive}. In other words, it should assign every location within
the area of interest (e.g. a region or a country) to one and only one class.
% This way they are ``building blocks''
A unit that is indivisable, consistent, and exhaustive can thus form the basis
of meaningful aggregation of space into spatial signatures.

% Existing options
The existing literature does not appear to have a satisfying candidate to act
as the building block of spatial signatures.
%
Without pretending to be exhaustive in the review, the vast majority of
existing approaches to delineate meaningful units of urban form and function
fall within one of the following three categories.
%% Administrative areas (cite Taubenbock 2020)
The first group relies on administrative units such as postcodes, census geographies
or municipal boundaries (e.g. \hl{XXXrefsXXX}).
%
These are practical as they usually are exhaustive of the geography and
readily available. However, their partition of space is usually driven by
different needs that rarely align with the measurement of spatial signatures,
or indeed those of any morphological or functional urban process. For example,
\cite{puente2020sensitive} compare the size distribution of urban areas in
Spain across several definitions and conclude they are fundamentally
different; similarly, \cite{taubenbock2019new} rank world's largest cities
based on their administrative size and on an alternative methods they proposed
following built-up area, reaching similar conclusions and even going on to
argue that ``\emph{administrative units obscure morphologic reality}''.
%% Uniform grids (cite papers on imagery for squared grids and mention H3)
%% Morphometric units --> MF to fill in here
% Not a criticism of these approaches, they're built for something else

% Proposal: the ET
%% One liner of what it is
%% Two foundational blocks:
%%% Enclosures (delimitters)
%%% Building (subdelimitters) --> use morphometric lit to justify why buildings are so relevant

%% Process of delineation:
%%% Gather all enclosing features (streets, rivers, railways, etc.)
%%% Build enclosing geography
%%% subdivision it based on building presence

% [Build illustration in the description]

% Why --> meets every requirement and extends the previous state-of-the-art
