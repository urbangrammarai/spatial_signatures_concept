\section{(Urban) form and function}
\label{sec:lit}

% Lit review backing up some of the claims made at the introduction (relevance etc)
% -	Discussing
% i.	Form: container
% ii.	Function: content
% iii.	The union between the two
% -	A bit on measurement
% i.	More thorough review of ways of measuring UFF


% Form
%% Form as a container
% - backup the idea of form as a container (lit)

%% Overview of attempts to describe form (time-based?, focus mostly on quant approaches)
% - the focus on the description of form is old but just recently becoming data driven

% - origins are in geography (Conzen and older) and architecture (Italians)

% - first data-driven attempts
% -- mention both morphometrics and remote sensing in parallel
% -- Hillier, Porta, Batty
% -- RS folks (land cover, urban/non-urban)

% - state of art
% -- overview of work in both umm and rs (data availability and ML as enablers)
% --- umm (Gil, Schirmer, Araldi, Berghauser Pont, Dibble) - focus on pure morphology
% --- RS (Taubenbock, Jochem, Kuffer, LCZ, UST) - needs a bit of work to get relevant literature
% --- The ways of measuring UF (link to EPB)
% --- touch on scale issue


% Function
%% Function as a content
% - geodem literature and econ urban spatial structure about the structure of employment and econ activity
% - Dani? (mostly geodem, economics etc.)
% --- touch on scale issue


% Union
%% The union between form and function
% - theory of the interconnectedness of the two
% - form <-> function

% - some papers combining both (from umm - Bourdic, Serra, partially Alexiou, some resilience work; add other)
% -- check Angel's papers
% -- productivity/econ, 
% -- efficiency
% -- environmental, fiscal (OECD) argument
% -- social integration and isolation
% -- accessibility

% --- touch on scale issue



% Limits (gaps)
% - limits of existing methods
% -- where these methods can't deliver
% --- detail, comprehensiveness, scalability (each lacks at least one)
% --- data requirements (some are dependent on detailed data (Berghauser Pont, high-res RS))
% ---- this could conclude with limits of open RS data "forcing" us to work with morphometrics -> link to the last part
% ---- we need training data before we can go RS way - we need theory before
% - limits of function
% - limits of existing existing work combining two
% opportunities to cross-pollinate to each other / hard for field to talk to each other
% SpSig should be unifying (link back to this section from section 3)
