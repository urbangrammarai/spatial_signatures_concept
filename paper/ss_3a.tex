\subsection{Definition}
\label{ssec:ss_def}

% short paragraph priming the SS idea and providng transition between sections 2 and 3




%%%%%%%
% link to existing similar concepts mentioned in the previous section
%% UST, LCZ, Urban Tissue, Morphological Region, geodem classification

SS sit in between purely morphological and purely functional.

characterisation of space based on form and function designed to understand urban environment
- geared towards understadning urban spaces
- SS - a distinct type of space based ...
- both granularity and scalability built-in

tissue def
A distinct area of a settlement in all three dimensions, characterised by a unique combination of streets, blocks/plot series, plots, buildings, structures and materials and usually the result of a distinct process of formation at a particular time or period.



%%%%%%% copy&paste from brainstorming
% Introduction of the concept of Spatial Signature as a building block reflecting physical, functional and socio-economical structure of cities.

%% atomic urban building block

%% multidimensional cross-sectional geodemographic unit

%% linkage of geodemographic concepts

%% cross-disciplinarity

%% spatial signatures should have the ability to classify all types of settlements and distinguish their physical and socio-economical differences (e.g. informal settlement vs slum)

%% why Spatial signature makes sense, how is it related to existing theories (e.g. city as a complex adaptive system)

% Goal, why SS?

% organic aggregation