\subsection{Definition}
\label{sec:ss_def}

% Definition
% What? Def
We propose the notion of \textit{spatial signatures} as:

\newtheorem*{theorem}{}
\begin{theorem}
A characterisation of space based on form and function designed to understand
urban environments
\end{theorem}

% Unpack
Spatial signatures provide exhaustive coverage for an area of interest by
drawing organic boundaries that delineate portions of consistent morphological
and functional characteristics.
%
We will refer to a single \textit{spatial signature} in two related but
distinct ways: first, as one of the multiple classes that make up a wider
typology of spatial signatures; and second, as a geographical instance of that
class, a contiguous portion of territory that shares those morphological and
functional traits.
%% The building block of cities
As such, spatial signatures can be seen as organically grown delineations that
organise space into urban and rural, orderly and irregular, formal and informal.
%
Laid out together, they can be used to explore urban extents, to parse through
the complexity of their spatial structure, or to understand the evolution of
cities.
%% Intermediate layer connecting purely morphological with functional (geo-dem)
In bringing together both form and function, with a focus on the urban,
spatial signatures provide a nexus between purely morphological
characterisations and those entirely based on function.
%
To the extent form and function are intrinsically connected, its
combination leads to more robust portraits of the space that makes up cities.
%% Focus on the urban: flip of traditional land-use/cover: the focus on the urban
And, since the focus is on the urban, spatial signatures provide a complementary
perspective to most land cover and use classifications, which historically
pay more attention to the portion of space not occupied by cities.
% Segway into the next two sections
Developing the concept of spatial signatures rests on two key pillars: the
spatial unit of choice and the embedding of urban form and function into such
units to be able to delineate signatures. We turn to both of these in more detail now.

%% SS as an organic aggregation of smaller building blocks that are similar and contiguous
%% Measurable

%--------------------------------------------------------------------------------------
% Original notes
%--------------------------------------------------------------------------------------

%%%%%%% copy&paste from brainstorming
%+ Introduction of the concept of Spatial Signature as a building block reflecting physical, functional and socio-economical structure of cities.

%+% atomic urban building block

%+% multidimensional cross-sectional geodemographic unit

%+% linkage of geodemographic concepts

%+% cross-disciplinarity

%+% spatial signatures should have the ability to classify all types of settlements and distinguish their physical and socio-economical differences (e.g. informal settlement vs slum)

%--------------------------------------------------------------------------------------
%+ Goal, why SS?

%+ organic aggregation

%+ it flips the land cover = urban is the main
% keep it short, one or two paragraphs

%  start with the definition (pretty technical one)
%characterisation of space based on form and function designed to understand urban environment
%+ geared towards understadning urban spaces
%+ SS - a distinct type of space based ...
%+ both granularity and scalability built-in

% a way of understanding space. not evena  unit
% information is the key component of it


%+ SS sit in between purely morphological and purely functional.



% urban tissue definition:  A distinct area of a settlement in all three dimensions,
% characterised by a unique combination of streets, blocks/plot series, plots,
% buildings, structures and materials and usually the result of a distinct process
% of formation at a particular time or period.



