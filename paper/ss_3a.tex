\subsection{Definition}
\label{ssec:ss_def}

% Definition
% What? Def
We propose the notion of \textit{Spatial Signatures} as:

\newtheorem*{theorem}{}
\begin{theorem}
A conceptualisation of space based on form and function designed to understand
urban envionments
\end{theorem}

% Unpack
Spatial Signatures provide exhaustive coverage for an area of interest by
drawing organic boundaries that delineate portions of consistent morphological
and functional characteristics.
%
We will refer to a single \textit{spatial signature} in two related but
distinct ways: first, as one of the multiple classes that make up a wider
typology of Spatial Signatures; and second, as a geographical instance of that
class, a contiguous portion of land that shares those morphological and
functional traits.
%% The building block of cities
As such, spatial signatures can be seen as organically grown delineations that
organise space into urban and rural, orderly and irregular, formal and informal.
%
Laid out together, they can be used to explore urban extents, to parse through
the complexity of spatial structure, or to understand the evolution of cities.
%% Granular, scalable, comparable
The concept of Spatial Signatures is granular but scalable; data-driven but
theoretically informed; and flexible enough to be adapted to a wide variety of
applied contexts, from data-rich to those with limited availability.
%% Intermediate layer connecting purely morphological with functional (geo-dem)
In bringing together both form and function, with a focus on the urban,
Spatial Signatures provide a nexus between purely morphological
characterisations, such as those of the morphometric literature; and those
entirely based on function, such as geodemographic classifications.
%
To the extent form and function are intrinsically connected, its
combination leads to richer portraits of the space that makes up cities.
%% Focus on the urban: flip of traditional land-use/cover: the focus on the urban
And, since the focus is on urban, Spatial Signatures provides a complementary
perspective to most land cover and use classifications, which historically
focus more intensely on the portion of space not occupied by cities.

% Why? Goals
%% Provide a theoretically-rich, data-driven approach to understand cities
%% Provide a shared vocabulary
%% Useful for many (``package urban form and function''), cross-disciplinary

% Give way to the following two sections
%% SS is a concept that can be implemented in many ways
%% SS as an organic aggregation of smaller building blocks that are similar and contiguous
%% Measurable
The following two sections cover both of these aspects in more detail.

%--------------------------------------------------------------------------------------
% Original notes
%--------------------------------------------------------------------------------------

%%%%%%% copy&paste from brainstorming
%+ Introduction of the concept of Spatial Signature as a building block reflecting physical, functional and socio-economical structure of cities.

%+% atomic urban building block

%+% multidimensional cross-sectional geodemographic unit

%+% linkage of geodemographic concepts

%+% cross-disciplinarity

%+% spatial signatures should have the ability to classify all types of settlements and distinguish their physical and socio-economical differences (e.g. informal settlement vs slum)

%--------------------------------------------------------------------------------------
%+ Goal, why SS?

%+ organic aggregation

%+ it flips the land cover = urban is the main
% keep it short, one or two paragraphs

%  start with the definition (pretty technical one)
%characterisation of space based on form and function designed to understand urban environment
%+ geared towards understadning urban spaces
%+ SS - a distinct type of space based ...
%+ both granularity and scalability built-in

% a way of understanding space. not evena  unit
% information is the key component of it


%+ SS sit in between purely morphological and purely functional.



% urban tissue definition:  A distinct area of a settlement in all three dimensions,
% characterised by a unique combination of streets, blocks/plot series, plots,
% buildings, structures and materials and usually the result of a distinct process
% of formation at a particular time or period.



