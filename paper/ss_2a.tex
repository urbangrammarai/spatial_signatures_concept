% Form
\subsection{Form}
\label{sec:lit_form}
%% Form as a container
% - backup the idea of form as a container (lit) TODO: introduction of the
%   section linking it back to the previous one

%% Overview of attempts to describe form (time-based?, focus mostly on quant
%approaches)
% - the focus on the description of form is old but just recently becoming data
%   driven
% - origins are in geography (Conzen and older) and architecture (Italians)
% - first data-driven attempts -- mention both morphometrics and remote sensing
%   in parallel -- Hillier, Porta, Batty -- RS folks (land cover,
%   urban/non-urban)
Urban form approaches environments from the perspective of their physical
structure and appearance. Research studying urban form has a long tradition,
dating back to the early 1900s \citep{geddes1915cities,
trewartha1934japanese}. Urban morphology, subsequently, begun in the 1960s as an
independent area of research. The field originated in parallel within geography
\citep{conzen1960alnwick} and architecture \citep{muratori1959studi}, reflecting
its inherently multi-disciplinary nature, later reinforced by the inclusion of
socio-economic elements, as in the work of \cite{panerai1997formes}. The
original methods are predominantly qualitative, a tendency that persists today
\citep{dibble2016urban}. The first notable quantitative approaches date to the
late 1980s and 1990s, reflecting advancements in computation and newly available
data capturing the built environment. In this context, two strains of research
have emerged. One focuses on cartographic (vector) representation of the urban
environment, assessing its boundaries \citep{batty1987}, street networks
\citep{hillier1996, porta2006} and other elements \citep{pivo1993taxonomy}. The
second one is based on earth observation, exploiting remotely sensed data to
capture change in the footprint of urban areas \citep{howarth1983landsat}.

% - state of art -- overview of work in both umm and rs (data availability and
%   ML as enablers) --- umm (Gil, Schirmer, Araldi, Berghauser Pont, Dibble) -
%   focus on pure morphology
The current state of the art still retains this distinction between cartographic
and remotely sensed approaches. A modern quantitative branch of urban
morphology, or urban morphometrics, has emerged working predominantly discrete
elements of urban form, and proposing an abundant selection of measurable
characters that describe different aspects of form
\citep{fleischmann2020measuring}. As part of this trend, methods focusing on a
single aspect \citep{porta2006} have been replaced by efforts to better reflect
the complexity of urban form through the combination of multiple morphometric
characters into a single model, often leading to data-driven typologies
\citep{song2007}. This focus on classification is becoming more prominent,
fueled by the possibilities afforded by new datasets increasingly available.
Indeed, the literature is now able to produce typologies that start from
small-scale studies focused on blocks and streets \citep{gil2012}, and zoom out
into larger areas with higher granularity \citep{schirmer2015, araldi2019,
bobkova2019, dibble2019origin, jochem2020}.

% --- RS (Taubenbock, Jochem, Kuffer, LCZ, UST) - needs a bit of work to get
% relevant literature
Advances in remote sensing have also led to a range of classification
frameworks based on various conceptualizations of the urban fabric. However,
there is one significant difference between classification derived via
morphometric characterization and the one based on remote sensing. Where the
former is mostly unsupervised \citep{araldi2019, schirmer2015}, exploiting the
hidden structure in the data to develop organically the typology; the latter
tends towards supervised techniques, relying on classes defined a priori
\citep{ pauleit2000assessing}. Two emerging classification models used to inform
these exercises are Local Climate Zones \citep{stewart2012}, defining ten
built-form types and seven land cover types, and used recently by
\cite{koc2017mapping} or \cite{taubenbock2020}; and the Urban Structural Type, a
generic typology based on the notion of internal homogeneity of types
\citep{lehner2019}.

% --- The ways of measuring UF (link to EPB) --- touch on scale issue QST: how
% detailed we want to have this? We may omit this.


