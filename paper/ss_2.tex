\section{(Urban) form and function}
\label{sec:lit}

% Lit review backing up some of the claims made at the introduction (relevance etc)
% -	Discussing
% i.	Form: container
% ii.	Function: content
% iii.	The union between the two
% -	A bit on measurement
% i.	More thorough review of ways of measuring UFF

% Form
\subsection{Form}
\label{sec:lit_form}
%% Form as a container
% - backup the idea of form as a container (lit) TODO: introduction of the
%   section linking it back to the previous one

%% Overview of attempts to describe form (time-based?, focus mostly on quant
%approaches)
% - the focus on the description of form is old but just recently becoming data
%   driven
% - origins are in geography (Conzen and older) and architecture (Italians)
% - first data-driven attempts -- mention both morphometrics and remote sensing
%   in parallel -- Hillier, Porta, Batty -- RS folks (land cover,
%   urban/non-urban)
Urban form approaches environments from the perspective of their physical
structure and appearance. Research studying urban form has a long tradition,
dating back to the early 1900s \citep{geddes1915cities,
trewartha1934japanese}. Urban morphology, subsequently, begun in the 1960s as an
independent area of research. The field originated in parallel within geography
\citep{conzen1960alnwick} and architecture \citep{muratori1959studi}, reflecting
its inherently multi-disciplinary nature, later reinforced by the inclusion of
socio-economic elements, as in the work of \cite{panerai1997formes}. The
original methods are predominantly qualitative, a tendency that persists today
\citep{dibble2016urban}. The first notable quantitative approaches date to the
late 1980s and 1990s, reflecting advancements in computation and newly available
data capturing the built environment. In this context, two strains of research
have emerged. One focuses on cartographic (vector) representation of the urban
environment, assessing its boundaries \citep{batty1987}, street networks
\citep{hillier1996, porta2006} and other elements \citep{pivo1993taxonomy}. The
second one is based on earth observation, exploiting remotely sensed data to
capture change in the footprint of urban areas \citep{howarth1983landsat}.

% - state of art -- overview of work in both umm and rs (data availability and
%   ML as enablers) --- umm (Gil, Schirmer, Araldi, Berghauser Pont, Dibble) -
%   focus on pure morphology
The current state of the art still retains this distinction between cartographic
and remotely sensed approaches. A modern quantitative branch of urban
morphology, or urban morphometrics, has emerged working predominantly discrete
elements of urban form, and proposing an abundant selection of measurable
characters that describe different aspects of form
\citep{fleischmann2020measuring}. As part of this trend, methods focusing on a
single aspect \citep{porta2006} have been replaced by efforts to better reflect
the complexity of urban form through the combination of multiple morphometric
characters into a single model, often leading to data-driven typologies
\citep{song2007}. This focus on classification is becoming more prominent,
fueled by the possibilities afforded by new datasets increasingly available.
Indeed, the literature is now able to produce typologies that start from
small-scale studies focused on blocks and streets \citep{gil2012}, and zoom out
into larger areas with higher granularity \citep{schirmer2015, araldi2019,
bobkova2019, dibble2019origin, jochem2020}.

% --- RS (Taubenbock, Jochem, Kuffer, LCZ, UST) - needs a bit of work to get
% relevant literature
Advances in remote sensing have also led to a range of classification
frameworks based on various conceptualizations of the urban fabric. However,
there is one significant difference between classification derived via
morphometric characterization and the one based on remote sensing. Where the
former is mostly unsupervised \citep{araldi2019, schirmer2015}, exploiting the
hidden structure in the data to develop organically the typology; the latter
tends towards supervised techniques, relying on classes defined a priori
\citep{ pauleit2000assessing}. Two emerging classification models used to inform
these exercises are Local Climate Zones \citep{stewart2012}, defining ten
built-form types and seven land cover types, and used recently by
\cite{koc2017mapping} or \cite{taubenbock2020}; and the Urban Structural Type, a
generic typology based on the notion of internal homogeneity of types
\citep{lehner2019}.

% --- The ways of measuring UF (link to EPB) --- touch on scale issue QST: how
% detailed we want to have this? We may omit this.




% Function
\subsection{Function}
\label{sec:lit_function}
%% Function as a content
% - geodem literature and econ urban spatial structure about the structure of employment and econ activity
% - Dani? (mostly geodem, economics etc.)
% --- touch on scale issue


% Definition: function is what
Urban function considers environments based on the activities that take place
within them.
%
The focus is thus not on what a space ``looks
like'', as it is the case on urban form, but on ``what it is used for''. What
activities occur within cities, how they are spatially configured, and how
they relate to each other are key questions in this context.
%
% The study of function is much more scattered across different literatures.
To the extent cities compress space and time to concentrate human activity of
very diverse nature, the study of function is relevant to a variety of
fields and is undertaken by a wider constituency of researchers. Disciplines
as disparate as geography, economics or environmental sciences, to name only a
few, have contributed in their own way to our understanding of urban
function.
% Policy
Furthermore, because function has direct implications for a wide range of
social and environmental processes at different geographic scales, their study
also falls within the realm of policy.
%
Given the breadth of perspectives and goals, research on urban form is
difficult to classify and a complete overview of its contributions is beyond
the scope of this paper. Instead, here we will highlight what we consider the
most relevant domains involved in the study of urban form: environmental
sciences, urban and public economics, urban and transport geography, and
sociology.

% Environmental and GI sciences: LU Vs LC
Environmental sciences have long considered urban function
in the context of the broader interest on understanding the natural
characteristics of the surface of the Earth.
%% GISc/EnvSci --> Land use (Vs cover)
An area that has attracted much effort relates to the development of
classifications of land cover and land use, the former describing the nature of
surfaces while the latter focusing on how those surfaces are used. Several
land cover classifications are available (e.g. CORINE,
        \citealp{europeanenvironmentagency1990}, in Europe;
the National Land Cover Database, \citealp{homer2012national}, in the US; or
the Land Cover CCI, \citealp{defourny2012land}, globally), as well as some,
although less, for land use (e.g. the Urban Atlas project, \citealp{urban_atlas}).
%
It is important to note that the distinction between cover and use in this
context is not clearcut and there is wide discussion around the relationship
between the two
(e.g. \citealp{fisher2005land, doi:10.1080/17474230802434187}).
This dichotomy resembles the more general one between form (cover) and
function (use).
% Urban Remote Sensing
While much of this research is not focused on urban environments, the urban
remote sensing community \citep{weng2018urban} is building a more explicit
bridge between these approaches and the study of cities. Such connection is
becoming possible thanks to methodological advances, including object-based image
analysis (OBIA, e.g. \citealp{prasad2015remotely}), machine
learning (e.g. \citealp{kuffer2016slums, georganos2018very, JOCHEM2018104})
or computer vision (e.g. \citealp{stark2020satellite}).
Taken altogether, this body of research is allowing us to rethink the extent
to which our understanding of function can in fact be inferred from form,
particularly in urban environments.

While land use/cover classifications attempt to characterise landscape
function in a broad way, many disciplines have developed more specific
interests in urban function.
%% Ecology
Sustainability studies, for example, are interested in how function is
configured within and across cities in so far as it relates to the level of
emissions \citep{angel2018shape} or energy consumption (XXXrefsXXX).
%%% Check refs on OECD sprawl report
% Social Sciences
The social sciences have a long-standing interest on the spatial
configuration of form because it affects several outcomes of prime interest.
Depending on the nature of these outcomes, form is conceptualised in one or
another way.
%% Urban economics --> density and agglomeration
Urban economics pays special attention to density of economic activity and, by
extension, of population \citep{ahlfeldt2019, duranton2020economics}, since
density is intimately related to theories of agglomeration, one of the
intellectual pillars of the field.
%% Public economics --> public finance sustainability
Although less central to its main goals, public economics is interested on the
configuration of urban function to the extent that it determines the
efficiency of certain public services provided by local governments (e.g.
XXXrefsXXX).
%% Sociology --> social integration/segregation
Sociologists have also found that different spatial configurations of function
over space is associated with different degrees of social mobility
\citep{ewing2016does} or socio-economic deprivation
\citep{venerandi2018scalable}.
%% Transport geography --> accessibility
More generally, transport researchers have built a robust body of knowledge
linking urban function and its spatial distribution to different degrees of
accessibility to jobs (XXXrefXXX) and amenities (XXXrefXXX), with clear
implications for socio-economic disparities.
%% There are more but these give a feel
These are some of the most relevant, but not all, connections that
researchers from a wide variety of backgrounds have drawn between urban
function, its location and different social, economic and environmental
outcomes.



\subsection{Existing gaps}
\label{sec:lit_gaps}
% Limits (gaps)
% - limits of existing methods
% -- where these methods can't deliver
% --- detail, comprehensiveness, scalability (each lacks at least one)
% --- data requirements (some are dependent on detailed data (Berghauser Pont, high-res RS))
% ---- this could conclude with limits of open RS data "forcing" us to work with morphometrics -> link to the last part
% ---- we need training data before we can go RS way - we need theory before
% - limits of function
% - limits of existing existing work combining two
% opportunities to cross-pollinate to each other / hard for field to talk to each other
% SpSig should be unifying (link back to this section from section 3)
However, all the methods above have certain limits, mostly related to detail, comprehensiveness and scalability, lacking at least on them. Detail reflects spatial granularity of resulting classification, where more granular, i.e. more detailed, unit has the ability to capture smaller nuances of the urban environment and better reflect local characters or a place. Methods based on a unit which can be further subdivided \citep{dibble2019origin,jochem2020,araldi2019,gil2012}, therefore does not ensure internal homogeneity, can result in classes driven by the heterogeneity instead of the unit instead of the actual pattern of urban form. Comprehensiveness refers to the number of characters (variables) used in the classification procedure. Small sets of characters as in \cite{bobkova2019} or \cite{serra2018a} are prone to a selection bias and will less likely reflect the complexity of the urban environment. Finally, scalability reflects the ability of the proposed method to scale up to large extents of metropolitan areas or national-level studies. While some works illustrate such a potential \citep{jochem2020, schirmer2015,bobkova2019,araldi2019}, others which may overcome other issues are less likely to scale from their original limits \citep{dibble2019origin}. Furthermore, computational scalability can be limited by data availability. Methods dependent on a high amount of detailed vector data \citep{bobkova2019} can be hardly applied in other contexts where such input is not available.

