\section{Conclusions}
\label{sec:conclusions}

As a characterization of space designed to understand the urban environment,
spatial signatures have the potential to provide a unique insight into the ways
human population creates and inhabits its cities. The illustrations are shown in the
the previous section indicate that the resulting patchworks of signatures are consequences
of the combination of a broad range of aspects stretching from topography, through
historical development to the current use of space, each influencing the nature of
the urban environment in its specific way.

The variability is also reflected in the changing scale and shape of the environment,
with very granular and organic patterns in medieval development, rigid grids of
industrial era and vast unbuilt natural areas limiting the expansion of nearby
cities. In this context, the enclosed tessellation shows a high degree of
the adaptability of the spatial unit based on the geographical position in which
cells are generated and consistently reflects the pattern of both built and
unbuilt areas.

Combination of form and function within a single classification method can tell
more than either of them would be able to do alone. Spatial Signatures therefore
bridge purely morphological approaches based on concepts like morphological
region (ref Oliveira's review), Local Climate Zone (ref) or Urban Structural
Type (Lehner) and functional classification as land use and land cover (Ref) or
mobility (ref) and population (ref), while building on their abilities to
comprehensively describe a single aspect of the environment.
%% I'd maybe mention, when you're talking of the LCZs, etc., that these are a
%bit % like a land use/cover but focused on urban, hence complimentary
That said, Spatial Signaures are complimentary... \martin{I need to find an
argument here later. My mind is blank at the moment.}

The proposed method provides a step forward in understanding the environment in
a detailed, granular manner. One which allows scalability to millions of
observations. In combination with an inclusive form-functional characterisation,
spatial signatures bring consistent classification which has implications for a
broad range of urban disciplines and enables their interoperability. When each of
the disciplines understands one of the components, it is easier to find a common
ground and interdisciplinary relations within a combined classification.

Spatial signatures can become a backbone for actions aiming under the umbrella of
Sustainable Development Goals, one which is data-informed and can be targeted
due to its inherent granularity. Furthermore, policy within the context of rapid
urbanization on the global South can benefit from classification which combines
form and function as development patterns and varying levels of their formality
are not always aligned with the functional aspects of such environments.

Spatial signatures are the way of thinking rather than a fixed method using the
exact set of variables. It is a way of conceptualising built (and non-built)
environment. The outputs from different regions or countries are different
manifestations of the same idea, leading to conceptual consistency rather than
a strictly numerical one. Specific regions may have specific needs regarding the
characterisation that need to be acknowledged in the applied method. Furthermore,
as illustrated in the previous section, data environments vary significantly and
we cannot assume the same quality all around the world. However, all such
adaptations are still fully within the conceptual notion of spatial signature.

Further research should look deeper into how signatures from different places
and time periods relate to each other and how they differ, to understand the
rules which are common and those which are geographically specific. Mirroring
Tobler's first law of geography (REF), similar cities tend to be located closer
to each other indicating the potential of identification of underlying
geographical patterns within signatures.

Moving forward, the current developments in remote sensing indicate the scope of
leveraging satellite imagery in the process of detection of signatures. Such a
method would then allow frequent temporal updates resulting in an AI-based
a form-function digital twin of the built environment.

% punchy end!
\martin{It is missing some closing short paragraph.}

%%% When you talk more forwardly towards the end, I'd maybe leave the bit on
%tech a bit more vague and focus on the conceptual benefits, opening up a)
%international and b) temporal comparisons. On the back of that, you can maybe
%hint the way to go to make those possible is through technological innovation
%such as satellite data and machine learning, but just leave it at that, you
%can't give away too much on the season trailer
