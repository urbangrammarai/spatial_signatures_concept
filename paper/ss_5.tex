\section{Conclusions}
\label{sec:conclusions}

% A bit more on what we do in the paper
This paper proposes the notion of spatial signatures as a characterisation of
form and function designed to understand urban spaces.
%
As such,
spatial signatures have the potential to provide unique insight into the ways
human populations create and inhabit cities.
%
Developing spatial signatures begins with a partition of space that is
theoretically aligned with their purpose. To this end, we propose the enclosed
tessellation. With appropriate spatial units at hand, we show how form and
function can be quantitatively built in.
%
Our contribution resides in the combination of a unifying approach to urban
form and function with the proposal of a new spatial unit (the ET) that is
theory-informed and data-driven.
%
In this respect, spatial signatures bridge
purely morphological approaches based on concepts like the morphological region
\citep{oliveira2020}, Local Climate Zone \citep{stewart2012} or Urban
Structural Type \citep{lehner2019}, with functional approaches such as land
use/land cover classifications \citep{georganos2018very} or mobility and
population \citep{gale2016creating}.

% Suggesting limitations
Rather than a particular technique or a rigid application, the spatial
signatures provide a \textit{way of thinking} about building detailed,
scalable and internally consistent characterisations of form and function in
cities. It is a way of conceptualising built (and natural) environment.
%
The outputs from different regions or countries can be understood as
different manifestations of similar concepts.
%
Such dissimilarities in themselves can be indication of unique
characteristics in the urban systems being compared.
%
In this context, the
spatial signatures can highlight and adapt to these cicumstances, while
retaining the intellectual consistency of a shared conceptual framework.

Differences in data availability between different regions of the
world currently preclude planetary-scale analysis that are fully consistent
and thus directly comparable. We do not see this as a limitation of the
conceptual framework we propose in this paper but one of current
data limitations. However, we believe this is a technical barrier that is
constantly being lowered by technological (e.g., new forms of satellite-based
data) and societal (e.g., user-generated databases such as OpenStreetMap)
advances in data generation. As these new forms of global datasets become
more and more sophisticated, the need for conceptual frameworks, such as the
spatial signatures, that provide theory-informed ways of leveraging them will
only increase.

The spatial signatures can be used by other researchers and policymakers
interested in cities, their form, and how activity is distributed within
them.
%
While this article provides the conceptual underpinnings of our proposal, we
also provide open-source software and documentation that can be freely used to
generate ET cells from a variety of widely available datasets, and to attach
form and function characters to those spatial units.
%
We envision this approach, and its outputs from particular
classifications, as a useful input to integrate urban form and function in
research across disciplines such as geography, planning, economics or
sociology.
%
For example, because the spatial signatures synthesize many datasets into an
intuitive, one-dimensional characterisation, they could be used in research
that links the spatial structure of cities to their degree of sustainability,
environmental performance, or economic productivity.
% SDGs and other applications
Similarly, since they operationalise conceptual ideas about how our current
cities can adapt to the main challenges of the century, spatial signatures
can play an important role in tracking progress on initiatives such as the
UN's Sustainable Development Goals.
%
Given the rapid urbanisation in the Global South, and the constant
retrofitting of cities in the Global North, developing consistent frameworks
to characterise cities and track their evolution has never been more
important.
%
We hope the present paper contributes in this direction and can be the seed
of further discussion and progress on these challenges.

% DAB: From Martin, decided to cut out
%Further research should look deeper into how signatures from different places and
%periods relate to each other and how they differ, to understand the rules which are
%common and those which are geographically specific. Mirroring Tobler's first law of
%geography \citep{tobler1970computer}, similar cities tend to be located closer to each
%other, indicating the potential of identification of underlying geographical patterns
%within signatures.

%Moving forward, the current developments in remote sensing indicate the scope of
%leveraging satellite imagery in the process of detection of signatures. Such a method
%would then allow frequent temporal updates resulting in an AI-based a form-function
%digital twin of the built environment.

