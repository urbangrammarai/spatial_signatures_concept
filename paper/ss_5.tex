\section{Conclusions}
\label{sec:conclusions}

% discussion - include note on grammar

%%%%%%%
between review and main contribution
% link to existing similar concepts mentioned in the previous section
%% UST, LCZ, Urban Tissue, Morphological Region, geodem classification

%% why Spatial signature makes sense, how is it related to existing theories
%% (e.g. city as a complex adaptive system)


%%%%%%
link back to section 3 and 4

then go back

form and function

context of the paper with intro


reinforcing the value (SDG etc)


%%% outline of the section

% As a characterization of space designed to understand urban environment,
% spatial signatures have a potential to provide a unique insight into the ways
% human population creates and inhabits its cities as illustrated
% in the previous section.

% The enclosed tessellation shows a high degree of adaptability of the spatial
% unit based on the context in which it is created with small granular in urban
% centres like in Barcelona and large open-space like in hills of Medellin.

% Combination of form and function can tell more than either of them would be
% able to do alone. Therefore it bridges morphological approaches like
% morphological region LCZ, or UST and geodem classification, while building on
% their ability to describe a single aspect of environment.
%% I'd maybe mention, when you're talking of the LCZs, etc., that these are a bit like a land use/cover but focused on urban, hence complimentary

% The method provides a step forward in understadning at granular scale and at
% scale. In combination with an inclusive form-functional characterisation,
% signatures have implication for a broad range of urban disciplines and enable
% their interoperabilty.

% SpSig can become a tool which influences actions within SDG
% framework, which can be data-informed and targeted due to inherent granularity

% Moving forward, there's a scope of leveraging satellite imagery in the process
% of detection of signatures to generate an up-to-date digital twin

% Spatial signatures are way of thinking - a way of conceptualising built (and
% non-built) environment. As such, the outputs from different regions or
% countries are different manifestations of the same idea.

% Further research can look into how signatures from different places relate to
% each other and how they differ (pointing towards urban grammar idea here)

%%% When you talk more forwardly towards the end, I'd maybe leave the bit on
%tech a bit more vague and focus on the conceptual benefits, opening up a)
%international and b) temporal comparisons. On the back of that, you can maybe
%hint the way to go to make those possible is through technological innovation
%such as satellite data and machine learning, but just leave it at that, you
%can't give away too much on the season trailer  