\section{Conclusions}
\label{sec:conclusions}

% A bit more on what we do in the paper
This paper proposes the notion of spatial signatures as a characterisation of
form and function designed to understand urban spaces.
%
As such,
spatial signatures have the potential to provide unique insight into the ways
human populations create and inhabit cities.
%
Developing spatial signatures begins with a partition of space that is
theoretically aligned with their purpose. To this end, we propose the enclosed
tessellation. With appropriate spatial units at hand, we show how form and
function can be quantitatively built in.
% Illustration
The illustration in the the previous section demonstrates how
the resulting patchwork of signatures reflects a broad range of aspects
stretching from topography to design, through history into the current use
of space, each influencing in its own way the nature of the urban fabric.

    % Contribution
% ET Cells
The flexibility of our approach is evidenced in the variety of scales and
shapes picked up in the analysis, and stems in great part from the
choice of the spatial unit we propose --the ET cell. Unlike other common
units, such as uniform grids or administrative boundaries, ET cells provide
full coverage while adapting to
varying conditions depending on both the urban and data relities of the
application at hand.
% Combination of F&F
Together with the use of the Enclosed Tessellation, the main contribution of
spatial signatures resides in the combination of form and function into a
single classification of space.
%
We believe this approach results in robust characterisations and is able to
provide more insights than the sum of those from each aspect alone.
In this respect, spatial signatures bridge
purely morphological approaches based on concepts like the morphological region
\citep{oliveira2020}, Local Climate Zone \citep{stewart2012} or Urban
Structural Type \citep{lehner2019}, with functional approaches such as land
use/land cover classifications \citep{georganos2018very} or mobility and
population \citep{gale2016creating}.
%
In doing so, they provide a complimentary view that adds a new perspective
rather than replaces existing classifications.

% Detailed, scalable and consistent
Rather than a particular technique or a rigid application, the Spatial
Signatures provide a \textit{way of thinking} about building detailed,
scalable and consistent characterisations of form and function in cities.
It is a way of conceptualising built (and non-built) environment. In this
context, the outputs from different regions or countries can be understood as
different manifestations of similar concepts. Specific
regions may present specific characteristics; and data landscapes vary
significantly across the globe, as our illustration shows. In both cases, the
spatial signatures can highlight and adapt to these cicumstances, retaining
the conceptual framework.

Such flexibility and intellectual malleability makes the spatial signatures
an excellent candidate to become a platform that brings together different
disciplines interested in cities, their form and how activity is distributed
within them. We envision this approach, and its outputs from particular
classifications, as a useful input to integrate urban form and function in
research across disciplines such as geography, planning, economics or
sociology.
% SDGs and other applications
Similarly, since they operationalise conceptual ideas about how our current
cities can adapt to the main challenges of the century, spatial signatures
can play an important role in tracking progress on initiatives such as the
UN's Sustainable Development Goals.
%
Given the rapid urbanisation in the Global South, and the constant
retrofitting of cities in the Global North, developing consistent frameworks
to characterise cities and track their evolution has never been more
important.
%
We hope the present paper contributes in this direction and can be the seed
of further discussion and progress on these challenges.

% DAB: From Martin, decided to cut out
%Further research should look deeper into how signatures from different places and
%periods relate to each other and how they differ, to understand the rules which are
%common and those which are geographically specific. Mirroring Tobler's first law of
%geography \citep{tobler1970computer}, similar cities tend to be located closer to each
%other, indicating the potential of identification of underlying geographical patterns
%within signatures.

%Moving forward, the current developments in remote sensing indicate the scope of
%leveraging satellite imagery in the process of detection of signatures. Such a method
%would then allow frequent temporal updates resulting in an AI-based a form-function
%digital twin of the built environment.

