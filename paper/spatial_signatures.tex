\documentclass[pdftex,letterpaper,11pt]{article}%
\usepackage[export]{adjustbox}
\usepackage{multirow}
\usepackage[arabicsections]{dpugatex}
%\usepackage{dpppl}
%\usepackage{lscape}
\usepackage{pdflscape}
\usepackage{longtable}

\usepackage{tikz}

\usepackage[british]{babel}
\usepackage{amssymb,amsmath,warpcol,url,ctable,multirow,caption,threeparttable,float,soul,gensymb}
%\usepackage[noend]{algpseudocode}
\makeatletter
\def\BState{\State\hskip-\ALG@thistlm}
\makeatother


\usepackage{subcaption}


\usepackage{xcolor,colortbl}%Per colorejar cel·les de les taules
\definecolor{verylightgray}{gray}{0.90}
\newcommand{\verylightgray}[1]{\cellcolor{verylightgray}#1}


\usepackage{arydshln} %Per dashed lines
\setlength\dashlinedash{1.5pt}
\setlength\dashlinegap{2.5pt}
\setlength\arrayrulewidth{0.8pt}

\usepackage[utf8]{inputenc}
\usepackage{eurosym}
\usepackage{soul}
\setstcolor{red} % Tatxat vermell pel text a eliminar!!!!
\definecolor{lightgreen}{rgb}{0.564706,0.933333,0.564706}
\newcommand{\miquel}[1]{{\sethlcolor{lightgreen} \hl{#1}} } % To highlight with light green Miquel's comments.
\soulregister\citet7 % These \soulregister commands allow us to avoid problems between soul functions (\hl, \st, ...) and these other functions (\citet,\citep,\ref,\footnote, ...). Include more if it is necessary.
\soulregister\citep7
\soulregister\cite7
\soulregister\ref7
\soulregister\footnote7
\soulregister\textit7
\soulregister\textbf7
\soulregister\textsc7
\soulregister\`7 % for open accents.
\soulregister\'7 % for closed accents.
\soulregister\% 7 % for \%age symbol
\usepackage{tabularx,tabulary}
\usepackage[osf]{mathpazo} % Amb aquest paquet faig ús de la lletra Palatino, la que fan servir Puga, Duranton et al. Amb l'opció [osf] la numeració és "old fashion", però és incompatible amb un títol en negreta i small caps (desapareixen les small caps!). Per tant, per ara opto per NO aplicar aquesta opció.
\usepackage[abs]{overpic}

\usepackage{hyperref}
\definecolor{darkblue}{rgb}{0.0,0.0,0.3}
\hypersetup{colorlinks=true, breaklinks=true, citecolor= darkblue, linkcolor=blue, urlcolor=blue}

\usepackage{chngcntr}

\newcommand{\mc}[3]{\multicolumn{#1}{#2}{#3}}
\newcommand{\mcc}[1]{\multicolumn{1}{c}{#1}}
\newcommand{\mcb}[1]{\multicolumn{1}{c}{\bf #1}}

\newcommand{\mr}[3]{\multirow{#1}{#2}{#3}} % Per fer Files múltiples (seguint el punt de vista de les columnes múltiples.
\newcommand{\mg}[3]{\multicolumn{#1}{#2}{\verylightgray #3}} %Combino el color de les cel·les amb al command de les columnes múltiples.



%\hypersetup{%
 % pdftitle={Race and neighborhoods in the 21$^{\text{st}}$ century},%
 % pdfauthor={Jorge De la Roca (NYU) , Ingrid Gould Ellen (NYU) and Katherine O'Regan (NYU)},%
 % pdfkeywords={race segregation, discrimination}}
%\pdfOpenFitWidth
%\pdfShowBookmarks

\onehalfspacing%
%\doublespacing%

\newenvironment{tablenote}[1]{\begin{list}{}{\vskip-5mm\relax
\setlength{\leftmargin}{#1} \setlength{\rightmargin}{\leftmargin}}
\item[]\footnotesize\vskip-7pt
{\em Notes}:\space\ignorespaces}{\end{list}}

\newcommand{\jdlradded}[1]{#1}
\newcommand{\dpadded}[1]{#1}
\newcommand{\jdlrdeleted}[1]{}
\newcommand{\dpdeleted}[1]{}
\newcommand{\jdlrcomment}[1]{}
\newcommand{\dpcomment}[1]{}
\newcommand{\comment}[1]{}
\newcommand{\martin}[1]{{\color{blue} Martin: [{#1}]}}
\newcommand{\dani}[1]{{\color{purple} Dani: [{#1}]}}

\begin{document}
\begin{titlepage}
\vspace*{1ex}
\begin{minipage}{\textwidth}
\begin{center}%

    {\textsb{\LARGE Spatial Signatures\\ \textit{Understanding (urban) spaces
    through form and function}}}\\[4ex]%


{\Large\textbf{Daniel Arribas-Bel}\footnote[1]{THANKS.}\footnote[2]{Geographic
Data Science Lab, Department of Geography and Planning, University of
Liverpool, Roxby Building , 74 Bedford St S , Liverpool , L69 7ZT, United
Kingdom}\footnote[3]{E-mail: \url{D.Arribas-Bel@liverpool.ac.uk}; phone: +44 (0)151 795 9727; website: \url{http://darribas.org}.
}}\\[1mm]
%
{\Large\textsb{Martin Fleischmann}\footnotemark[1]\footnotemark[2]\footnote[4]{
E-mail: \url{M.Fleischmann@liverpool.ac.uk}; website: \url{https://martinfleischmann.net/}.}}\\[1mm]
{\large\textit{Geographic Data Science Lab, University of Liverpool} }\\[2.5ex]
%
July 2021\vspace{1.5ex}
%
\end{center}
%
\begin{abstract}
% Punchy intro
        This paper presents the notion of spatial signatures as a
        characterisation of space based on form and function designed to
        understand urban environments.
% Background - What
The spatial organisation of the different components of cities is relevant
for at least two main reasons.
%
On
the one hand, it encodes many aspects of the phenomena
that created such an arrangement in the first place. On the other, once in
place, this arrangement of urban form and function underpins many
outcomes, from economic productivity to environmental sustainability.
% Proposal - How
Our approach unfolds in three main stages.
%% ET
First, we propose a new spatial unit --the Enclosed Tessellation (ET) cell-- to
delineate space in a way that is exhaustive and matches the underlying
processes at which urban form and function operate.
%% Build F&F
Second, we propose to attach a large variety of form and function-based
characters to ET cells to describe each of these units.
%% Clustering
Third, to build spatial signatures, information on ET cells can be clustered
using unsupervised learning techniques.
%% Result
This process results in a theory-informed, data-driven
typology of space that follows form and function.
% Illustration
We illustrate this approach by applying it to a sample of five very different
cities scattered across the world. Our results demonstrate the ability to
successfully differentiate different parts of a city that were built at
different points in time and under different technological regimes, but also
highlight broader comparisons about the nature of urban fabric in different
regions.
% Benefits - Why
Our contribution resides in leveraging modern data, technology and methods to propose a
detailed, consistent and scalable methodology that characterises urban form and
function.
%
The spatial signatures can be used across academic disciplines and by a variety of
practitioners and policymakers supporting initiatives such as the Sustainable
Development Goals.
\end{abstract}
%
\vspace{1.5ex}
%
Key words: \hskip.25em Geographic Data Science, Urban Form, Urban Function\\
%
\vspace*{-1.5ex}
%
\end{minipage}
\end{titlepage}

% Section 1 - Intro
\section{Introduction}
\label{sec:intro}

% Importance of cities
% Importance of the spatial arrangement of cities
How we spatially arrange the building blocks that make up a city matters.
%
The map of many European cities tells the story of the different historical
periods in which they were born, grew and, in some cases, contracted.
%
American cities which ``came of age'' in the second half of the 20th Century
would look very different had the automobile not been the defining
technology of the time \citep{jacobs2016death}.
%
And the stark contrasts between luxury developments and informal settlements
that can be observed across many cities of the Global South are a reflection
of the wide range of disparities and inequalities that those societies
display \citep{alsayyad2003urban}.
%
This encoding of history, technology, and culture is sticky: once in place,
elements of the urban fabric change slowly over time. Although cities are constantly in flux, new
innovations and waves of change rarely start from scratch. More commonly, they
are added in a layered way. Over time, each phase, each change blends in with the rest
of the urban fabric to give a city its uniquely distinct pattern that defines
it almost as a strand of DNA.
% Two sides of arranging stuff: form and function
The building blocks of this process include the different
elements of the built and natural environments of which cities are composed,
but also the purpose they serve.
Understanding the former thus requires us to consider urban \textit{form}, while
grasping the latter invites us to examine its \textit{function}.
% FF encode history, technology and the values of the society that shapes
% them
Urban form and function are relevant for two main reasons. First, their fabric
encodes the socio-economic history, technology and values
of the society that has built them.
%
Studying the nature and distribution
of form and function in cities thus helps us better understand the societies
that, over time, have shaped them.
% but, once in place, it also has important implications
Second, urban form and function are
not only a history book recording the past, but also play an important role in
defining the present and shaping the future.
%
Once in place, their features and structure have direct implications for a
wide range of outcomes, from productivity and job access to social inclusion
and mobility, deprivation, service provision, energy consumption or carbon
emissions, to name just a few.
% 
In this context, the main contribution of the present paper is to
introduce the concept of spatial signatures as a characterisation of
space based on form and function designed to understand urban environments.
Our proposal advances contributes to the literature by filling a series of
existing gaps, including conceptual fragmentation and lack of detailed
evidence at scale.

% Gaps
The study of urban form and function is deeply fragmented
\citep{kropf2014ambiguity,brenner2015towards,gauthier2006mapping}.
%% fragmentation (academia/policy, disciplines)
Work with this focus is scattered across
different academic disciplines and policy-making scenes. This is not
necessarily a problem in itself since different backgrounds
provide a richer picture. And there is much to be gained from a plurality of
perspectives.
%
It does however mean that the evidence available presents different
interests as well as varying degrees of detail, consistency, and coverage.
% Academia
It is understandable that economists develop conceptualisations shaped around
economic theories (eg. \citealp{ahlfeldt2019}), while geographers do so paying attention
to spatial scales (eg. \citealp{boeing2018}),
and yet other disciplines bring different aspects to focus.
% Policy
Similarly, decision-makers interested in understanding aspects of urban form
and function tend to see it through the lenses provided by the vantage point
they occupy. Regional planners may try to obtain as much detail as
possible for a relatively small geographical area (eg. \citealp{bcnnight});
while supra-national organisations may prioritise scale and coverage
at the expense of detail (eg. \citealp{eea2016}).

%% need for detailed, consistent and scalable: currently, pick two
There is a clear need for detailed, consistent, and scalable evidence on urban
form and function.
% Why
Granular measurement that can be performed across large
geographical extents in a comparable way can unlock insights that get lost
when we only consider their parts in isolation. Similarly, analysis at scale
that lack sufficient detail also miss important aspects.
%
This is because many of the
theoretical underpinnings of urban form and function that reflect its history
and influence present and future outcomes tend to operate at fine scales; but,
to be able to observe meaningful differences, we need to consider many and
different places.
%
For example, the characters that define Medieval city centres in Europe
quickly blur when the geographical unit considered is coarse. But, to be able to
examine how these characters relate to different levels of walkability, or
even of gas emissions, we require a large extent to set up meaningful
comparisons.

% What we currently have
Of detail, consistency and scale, the current research landscape described above
is able to provide, at best, any combination of pairs from those three \citep{jochem2020,araldi2019,fleischmann2021methodological}.
%
There is an abundance of detailed studies that measure form and
function in an internally consistent way, but these are in their majority confined to
case studies with very limited geographical extents.
%
On the other end of the scale spectrum, recent years have seen the appearance
of work at a global scale that is internally consistent. However, their degree of detail
tends to be hindered by data limitations.
%
Finally, one could understand the multitude of detailed case studies in
conjunction as a growing body of evidence that is able to reach a sizeable
scale. But, in these case, the fragmentation discussed earlier often
translates in a lack of consistency that prevents meaningful comparisons.

% Opportunities
Recent advances in data, technology, and methodology are beginning to overcome
these limitations.
%% data
New forms of data such as open cadastres, consumer datasets derived from
modern business operations, or high resolution, public satellite imagery are
greatly improving the descriptions we can build of cities
\citep{arribas2014accidental, glaeser2018big, wei2020multiscale, fleischmann2021evolution}. Progressively, we are able to build denser and
more up to date representations of urban environments at a cheaper cost.
%% technology
Technological developments such as the dramatic increase of computational
power available to researchers, or improvements in computer algorithms and machine
learning are lowering the entry barrier to advances that only a few years ago
required a high degree of dedicated expertise to be able to benefit from.
%% methodologies (morphometrics, GDS)
Perhaps more importantly, recent methodological contributions such as
morphometrics \citep{dibble2016urban} or geographic data science
\citep{singleton2021geographic} are paving the way to blend these advances
with domain knowledge and urban theory, effectively enabling disciplines
concerned with the form and function of cities to benefit from such
developments.

% Preview
%% What: form and function
The spatial signatures are thus a delineation that divides geographical space
based on its appearance (form) and how it is used (function).
It is not a classification of space as much as a way of thinking around
classifying space based on form and function.
%%
The concept is data-driven but theoretically informed;
granular but scalable; and flexible enough to be adapted to a wide variety of
applied contexts, from data-rich to those with limited availability.
%
%% Provide a theoretically-rich, data-driven approach to understand cities
Spatial signatures embed theoretical ideas about how cities are
spatially arranged, how this configuration can be best conceptualised, and how
it is perceived by humans into a data-driven framework that connects them to the
vast amount of empirical information available representing the world.
%
%% Useful for many (``package urban form and function''), cross-disciplinary
These theoretical underpinnings are sourced from a variety of disciplines,
from architecture to environmental sciences, and thus are inherently
interdisciplinary.
%
%% Provide a shared vocabulary
The spatial signatures thus provide a shared vocabulary to bring together a
variety of scholars and policy makers for whom form and function in cities is
relevant, either as their main object of study or as an input for their own domains
of expertise.
%
%%
These characteristics make the spatial signatures an ideal candidate
for deployment on a wide range of data landscapes and geographical regions.
To demonstrate this flexibility, Section \ref{sec:app} presents five illustrations
that each apply the concept of spatial signatures to a different city, developing
internally consistent classifications for each of them.
%
It is important to emphasize we view these illustrations as a way to empirically
showcase our conceptual proposal rather than as the empirical, main contribution of
the paper.

% Relevance
In part because of their adaptability,
the spatial signatures hold important potential both for urban academics and
policy-makers.
%% Academic
From an academic point of view, they are relevant as a goal in themselves
that allows us to better measure and study the spatial configuration of the
building blocks that make up cities. But they also represent a platform on
which other disciplines can build on to embed form and function on a variety
of socio-environmental outcomes.
%% Policy
For policy-makers, the spatial signatures provide a framework for detailed
spatial understanding of the cities and territories their decisions affect.
They are useful both in the global
north, where cities are constantly recast and retrofitted, as well as the
global south, where most of the new urbanisation is currently taking place.
%
In summary, the spatial signatures allow us to move forward in realising
detailed, consistent, and scalable measurement of form and function in
cities.

% Remainder structure of the paper
The remaining of the paper is structured as follows. Section \ref{sec:lit}
reviews existing literature on urban form and function, highlighting current
gaps. Section \ref{sec:ss} represents the core of our contribution by detailing
our proposal of spatial signatures,
including how we define them, the spatial unit we develop to
measure them --the enclosed tessellation cell--, and the embedding of form and
function into such unit. In Section \ref{sec:app}, we illustrate the
flexibility of the spatial signatures by presenting five illustrations of the
spatial signatures to rather different global cities. And we conclude in Section
\ref{sec:conclusions} with some reflection about the value and potential of
our approach.



%--------------------------------------------------
% Bits to make sure to mention
%--------------------------------------------------



%+ These needs relate to both developing world, where there is not data at all
%  and most of the changes; but also to the developed world where life is being
%  "recast" and cities continue to evolve (housing crisis, remote work, climate
%  change targets, technology, etc.)


%+ SS are fine-grained, consistent
%+ SS data-driven but theoretically informed; granular but scalable;and flexible enough to be adapted to a wide variety of applied contexts
%x SS are intellectually "all-ecompassing", get around fragmentation
%+ need for detailed, consistent and scalable evidence (pick two of those)
%+ Fragmented literature --> fragmented measurement --> fragmented evidence
%+ Fragmentation hinders our understanding


% Section 2 - FF
\section{(Urban) form and function}
\label{sec:lit}

% Lit review backing up some of the claims made at the introduction (relevance etc)
% -	Discussing
% i.	Form: container
% ii.	Function: content
% iii.	The union between the two
% -	A bit on measurement
% i.	More thorough review of ways of measuring UFF


% Form
%% Form as a container
% - backup the idea of form as a container (lit)
% TODO: introduction of the section linking it back to the previous one

%% Overview of attempts to describe form (time-based?, focus mostly on quant approaches)
% - the focus on the description of form is old but just recently becoming data driven
% - origins are in geography (Conzen and older) and architecture (Italians)
% - first data-driven attempts
% -- mention both morphometrics and remote sensing in parallel
% -- Hillier, Porta, Batty
% -- RS folks (land cover, urban/non-urban)
Research studying urban form has a long tradition \citep{geddes1915cities, trewartha1934japanese}, whilst urban morphology as an independent area of research has established in the 1960s. It originated independently in geography \citep{conzen1960alnwick} and architecture \citep{muratori1959studi}, reflecting its inherently multi-disciplinary nature, which was later reinforced by the inclusion of socio-economic component in works of \cite{panerai1997formes}. The original methods are predominantly qualitative, and this tendency persists \citep{dibble2016urban}. First notable quantitative approaches date to the late 1980s and 1990s, reflecting advancements in computer science and newly available data capturing built environment. Two strains of research have emerged, one based on cartographic (vector) representation of the urban environment, assessing its boundaries \citep{batty1987}, street networks \citep{hillier1996, porta2006} and other elements \citep{pivo1993taxonomy}. The other one based on earth observation exploiting remotely sensed data to capture the change of footprints of urban areas \citep{howarth1983landsat} or classification of land cover \citep{europeanenvironmentagency1990}. 

% - state of art
% -- overview of work in both umm and rs (data availability and ML as enablers)
% --- umm (Gil, Schirmer, Araldi, Berghauser Pont, Dibble) - focus on pure morphology
The distinction between two approaches based on their primary source of data can also be applied to the recent literature reflecting the state of the art of characterization of urban form. A quantitative branch of urban morphology, or urban morphometrics, working predominantly with vector representation of elements of urban form has rapidly grown and offers an abundant selection of measurable characters describing different aspects of form \citep{fleischmann2020measuring}. Methods focusing on a single aspect \citep{porta2006} were replaced by others aiming to better reflect the complexity of urban form by combining multiple morphometric characters into a single model, often leading to a classification of some sort \citep{song2007}. The focus on classification is becoming more present in recent years, starting from small-scale studies classifying blocks and streets \citep{gil2012} to larger areas and higher granularity \citep{schirmer2015, araldi2019, bobkova2019, dibble2019origin, jochem2020}.

% --- RS (Taubenbock, Jochem, Kuffer, LCZ, UST) - needs a bit of work to get relevant literature
In parallel, advancements in remote sensing led to a range of classification frameworks based on various conceptualizations of the urban fabric. However, there is one significant difference between classification derived via morphometric characterization and the one of remote sensing origin. Where the former is mostly unsupervised \citep{araldi2019, schirmer2015}, the latter tends towards supervised techniques, capturing classes defined prior to the analysis \citep{ pauleit2000assessing}. Two most prominent classification models used as an input (i.e. training set) are Local Climate Zones \citep{stewart2012} defining ten built-form types and seven land cover types, used by \cite{koc2017mapping} or \cite{taubenbock2020}, and Urban Structural Type, a generic typology based on the notion of internal homogeneity of types \citep{lehner2019}. 

% --- The ways of measuring UF (link to EPB)
% --- touch on scale issue
% QST: how detailed we want to have this? We may omit this.


% Function
%% Function as a content
% - geodem literature and econ urban spatial structure about the structure of employment and econ activity
% - Dani? (mostly geodem, economics etc.)
% --- touch on scale issue


% Union
%% The union between form and function
% - theory of the interconnectedness of the two
% - form <-> function

% - some papers combining both (from umm - Bourdic, Serra, partially Alexiou, some resilience work; add other)
% -- check Angel's papers
% -- productivity/econ, 
% -- efficiency
% -- environmental, fiscal (OECD) argument
% -- social integration and isolation
% -- accessibility

% -- papers coming from morphological background (mostly)
% Bourdic, Salat and Nowacki 2012 - Land use, mobility, water, biodiversity, equity, economy, waste, culture, energy
% Serra, Psarra and O'Brien 2018 - Income deprivation (IMD)
% Song and Knaap 2007; Song, Popkin and Larsen 2013 - Land use, accessibility, transport
% Torrens 2008 - occupier profile
% Zheng et al 2014 - population density, proximity
% Ewing 2002 (SGA) - population (county level)
% Venerandi 2019 - social housing, commerce & services, job accessibility, housing stock
% Alexiou 2016 - proximity, very little form

% OECD 2018 - top level focus on sprawl, mostly population

% The link between the two (f+f) seems to be very weak and not many people combine both into a single classification method. Furthemore, if they do, the one or the other is just very simplified (population only, density only). Alternatively, they combine them reasonably (Bourdic 2012), but the result lacks granularity (their work is on city scale, Ewing on county scale).

% --- touch on scale issue



% Limits (gaps)
% - limits of existing methods
% -- where these methods can't deliver
% --- detail, comprehensiveness, scalability (each lacks at least one)
% --- data requirements (some are dependent on detailed data (Berghauser Pont, high-res RS))
% ---- this could conclude with limits of open RS data "forcing" us to work with morphometrics -> link to the last part
% ---- we need training data before we can go RS way - we need theory before
% - limits of function
% - limits of existing existing work combining two
% opportunities to cross-pollinate to each other / hard for field to talk to each other
% SpSig should be unifying (link back to this section from section 3)
However, all the methods above have certain limits, mostly related to detail, comprehensiveness and scalability, lacking at least on them. Detail reflects spatial granularity of resulting classification, where more granular, i.e. more detailed, unit has the ability to capture smaller nuances of the urban environment and better reflect local characters or a place. Methods based on a unit which can be further subdivided \citep{dibble2019origin,jochem2020,araldi2019,gil2012}, therefore does not ensure internal homogeneity, can result in classes driven by the heterogeneity instead of the unit instead of the actual pattern of urban form. Comprehensiveness refers to the number of characters (variables) used in the classification procedure. Small sets of characters as in \cite{bobkova2019} or \cite{serra2018a} are prone to a selection bias and will less likely reflect the complexity of the urban environment. Finally, scalability reflects the ability of the proposed method to scale up to large extents of metropolitan areas or national-level studies. While some works illustrate such a potential \citep{jochem2020, schirmer2015,bobkova2019,araldi2019}, others which may overcome other issues are less likely to scale from their original limits \citep{dibble2019origin}. Furthermore, computational scalability can be limited by data availability. Methods dependent on a high amount of detailed vector data \citep{bobkova2019} can be hardly applied in other contexts where such input is not available.



% Section 3 - SS
\section{Spatial Signatures}
\label{sec:ss}

% Benefits of blended FF
Despite the current sparsity of studies, we believe there are several benefits
in considering form and function in tandem when trying to understand urban
spaces.
% Gap for blended FF
%% FF are deeply interconnected, and some insights require the two
The two are deeply interconnected. This close
correlation implies that outcomes observed across form tend to hold true for
function, and viceversa. However, unique patterns emerge when particular types
of form and function come together. 
%% It is the combination of the two that encodes history, cultury, technology, etc.
We argue that it is only through the combination of form and function
that cities are able to encode and reflect sophisticated aspects of human
nature such as history, culture or technology.
%
In these cases, considering only one or the other hinders rather than enables,
as we risk missing important traits of the nature of a place.
%% From a pragmatic perspective, considering the two feeds more information, which leads to more robust representations
From a more empirical perspective, even when the two dimensions mostly
overlap, there is value in considering them jointly. Some aspects of form and
function are clear conceptually but challenging to measure. Broadening the
pool of indices that can be deployed ensures better accuracy when
characterising existing patterns on the ground.
%
In this section, we detail our proposal to understand urban form and function
through what we term “Spatial Signatures”. 



\subsection{Definition}
\label{ssec:ss_def}

% keep it short, one or two paragraphs

%  start with the definition (pretty technical one)
characterisation of space based on form and function designed to understand urban environment
- geared towards understadning urban spaces
- SS - a distinct type of space based ...
- both granularity and scalability built-in

% a way of understanding space. not evena  unit
% information is the key component of it


% SS sit in between purely morphological and purely functional.



% urban tissue definition:  A distinct area of a settlement in all three dimensions,
% characterised by a unique combination of streets, blocks/plot series, plots,
% buildings, structures and materials and usually the result of a distinct process
% of formation at a particular time or period.



%%%%%%% copy&paste from brainstorming
% Introduction of the concept of Spatial Signature as a building block reflecting physical, functional and socio-economical structure of cities.

%% atomic urban building block

%% multidimensional cross-sectional geodemographic unit

%% linkage of geodemographic concepts

%% cross-disciplinarity

%% spatial signatures should have the ability to classify all types of settlements and distinguish their physical and socio-economical differences (e.g. informal settlement vs slum)

% Goal, why SS?

% organic aggregation

% it flips the land cover = urban is the main

\subsection{Building blocks: the Enclosed Tessellation}
\label{ssec:ss_et}

% What
This section proposes a novel and theoretically-informed delineation of space to
support the development of spatial signatures.
% Why % Spatial unit is a big deal
Since spatial signatures are conceptualised as highly granular in space,
considering the ideal unit of analysis at which to measure them is of utmost
importance.
%
This step is worth spending energy and effort for two main reasons.
%% If improper --> MAUP big time
First, if ignored, there is an important risk of incurring the modifiable areal
unit problem (MAUP, \citealp{openshaw1981modifiable}). The urban fabric is not a
spatially smooth phenomenon; rather, it is lumpy, irregular and operates at very
small scales.
%
Choosing a spatial unit that does not closely match its distribution will
subsume interesting variation and will hide features that are at the very heart
of what we are trying to capture with spatial signatures.
%% If done right --> a direct way to learn about geography
Second, and conversely, we see adopting a meaningful unit a step of analysis
itself. Rather than selecting an imperfect but existing unit to try to
characterise spatial signatures, delineating our own is an opportunity in itself
to learn about the nature of urban tissue and better understand issues about
distribution and composition within urban areas.

% How: Requirements of a good unit
Let us first focus on what is required from an ideal unit of analysis for
spatial signatures. We need a partition of space into sections of built
\textit{and} lived environment that can later be pieced together based on their
characteristics. The result will feed into an organic delineation that captures
variation in the appearance and character of urban fabric as it unfolds over
space.
%
To be more specific, a successful candidate for this task will need to fulfill
at least three features: indivisibility, internal consistency, and exhaustivity.
%% Undivisable: like the atom for urban fabric
An ideal unit will need to be \textit{indivisable} in the sense that if it were
to be broken into smaller components, none of them would be enough to capture
the notion of spatial signature.
%% Internally consistent: contain only one type of fabric
Similarly, every unit needs to be \textit{internally consistent}: one and only
one type of signature should be represented in each observation.
%% Exhaustive: every portion of a geography should be covered and assigned into
%a single unit
Finally, the resulting delineation needs to be geographically
\textit{exhaustive}. In other words, it should assign every location within the
area of interest (e.g. a region or a country) to one and only one class.

% Existing options
The existing literature does not appear to have a satisfying candidate to act as
the building block of spatial signatures.
%
Without attempting an exhaustive review, an endeavour beyond the scope of this
article, the vast majority of existing approaches to delineate meaningful units
of urban form and function fall within one of the following three categories.
%% Administrative areas (cite Taubenbock 2020)
The first group relies on \textit{administrative} units such as postcodes,
census geographies or municipal boundaries (e.g. \citealp{taubenbock2020}).
%
These are practical as they usually are readily
available. However, their partition of space is usually driven by different
needs that rarely align with the measurement of spatial signatures, or indeed
those of any morphological or functional urban process.
\cite{taubenbock2019new} even argue that
``administrative units obscure morphologic reality''.
%% Uniform grids (cite papers on imagery for squared grids and mention H3)
An emerging body of work relies on granular, \textit{uniform grids} as the main
unit of analysis (e.g. \citealp{jochem2020}). This choice is usually explicitly or implicitly motivated
by the lack of a better, bespoke partitioning; the use of input data distributed
in grids (e.g. satellite imagery); and the assumption that, with enough
resolution, grids can be organically aggregated into units that match the
processes of interest.
%% Morphometric units --> MF to fill in here
A third approach followed mostly by the literature on urban morphology relies on
the definition of morphometric units. These include street segments
\citep{araldi2019}, plots \citep{bobkova2019}, building footprints
\citep{schirmer2015}, or constructs such as the Sanctuary area
\citep{mehaffy2010urban,dibble2019origin}.
% Not a criticism of these approaches, they're built for something else
In all these cases, the
choice is justified by the particular application in which it takes place.
However none of these approaches meet the three characteristics we require for
spatial signatures.
%
Administrative boundaries are exhaustive but rarely indivisable or consistent
when it comes to urban form, usually grouping very different types of fabric
within a single area.
%
Uniform grids are also exhaustive but
the arbitrariety of their delineation with respect to urban form leaves
them divisable and internally inconsistent. Even high resolution
grids (i.e., very granular cells) might not be an ideal approach to capture
the variability of urban form and function over space.\footnote{For example,
areas with little variation in form and function, such as large parks or
natural spaces, would be overrepresented in that it would take many cells to
cover a single morpho-functional unit. At the same time, a single cell in
areas with more intense variation of form and function, such as city centres,
would aggregate more than one unit into a single cell. At the heart of this behaviour is the
fact that uniform grids account for space in a directly proportional way to
their area. This is not how form and function unfolds spatially.}
%
Morphometric units are the most theoretically appealing ones since they are built to
match the distribution of urban features and are usually granular enough to
warrant internal consistency and indivisibility. Most of them are however not
exhaustive as they are anchored to particular elements of the built environment,
such as streets or building footprints, which do not provide full coverage.
Plots would theoretically meet all characteristics but can be problematic due to
their variable definition leading to different geometric representations
\citep{kropf2018plots}.

% Proposal: the ET
We propose the development of a new spatial unit that we term the
\textit{enclosed tessellation cell} (EC).
%% One liner of what it is
An EC is defined as:
\begin{theorem}
        The portion of space that results from growing a morphological
tessellation within an enclosure delineated by a series of natural or built
barriers identified from the literature on urban form, function and perception.
\end{theorem}

\begin{figure}
\includegraphics[width=\linewidth]{figures/et_diagram.pdf}
\caption{Diagram illustrating the sequential steps leading to the delineation of
enclosed tessellaion. From a series of enclosing components, where blue are streets and
yellow river banks (A), to enclosures (shown as blue polygons) (B),
incorporation of buildings as anchors (C) to final tessellation cells shown as red polygons (D).}
\label{fig:et_diagram}
\end{figure}

%% Process of delineation:
Let us unpack this concept a bit further. ET cells result from the combination of
three sequential steps (Figure \ref{fig:et_diagram}).
%%% Gather all enclosing features (streets, rivers, railways, etc.)
First, they rely on a set of enclosing components: features of the landscape
that divide it in smaller, fully delimited portions. The list of what should be
counted as enclosing is informed by theory and, as we will see below, may vary
by context. But, as an illustration, it includes elements such as the street
network, rivers and coastlines, or railways.
%%% Build enclosing geography
Second, these enclosing features are integrated into a single set of boundaries
that partition the geography into smaller areas. In some cases, they will be
small, as with urban blocks in dense city centres; in others, they will be
larger in size, as in rural sections with lower density of enclosing features.
We call each of this fully delimited areas an enclosure.
%%% subdivision it based on building presence
Third, enclosures are further subdivided using a morphological tessellation
\citep{fleischmann2020morphological}, a method derived from Voronoi tessellation taking polygons instead of points as an input,
that exhaustively partitions space based on a set of building footprints,
which are used in this context as anchors to draw catchment polygons.
%%
This three-step process generates geographical boundaries for a given area that result in a
new spatial unit. This unit provides full geographical coverage without any
overlap.
%
Since the essence of the approach resides in growing a tessellation inside a set
of enclosing features, we call the resulting areas ``enclosed tessellation
cells''.

%% Two foundational blocks:
The enclosed tessellation (ET) intersects two perspectives of how space can be
understood and organised.
%
%%% Enclosures (delimitters)
The first relies on the use of features that \textit{delimit} the landscape and
partition it into smaller, fully enclosed portions. These include the road and
street networks, but also others such as railways or rivers. Each feature is
conceptualised as a line that acts as a boundary, dividing space into what falls
within each of its sides.
%
A long tradition in the literature on urban perception relies on
variations of these delimiters. Prominent
early examples include the edges and paths highlighted by \cite{lynch1960} as
two of the five core elements that define legibility and imageability of a city;
as well as the later work inspired by this framework (e.g. \citealp{filomena2019a}).

%%% Building (subdelimitters) --> use morphometric lit to justify why buildings
%are so relevant
The second perspective that ET integrate is a vision organised around
\textit{anchors}. In this view, space arises in-between
a discrete set of relevant features. Unlike delimiters, these elements do not
partition space per se, but instead act as origins to which the rest can be
``attached''.
%
The choice of anchors might vary by context but, in this case, the literature on
morphometrics has extensive evidence to support the use of buildings as the
primary feature \citep{hamaina2012a, usui2013estimation, schirmer2015}.

% [Build illustration in the description]

% Why --> meets every requirement and extends the previous state-of-the-art
The combination of delimiters and anchors as the parsers of space make ET cells an
ideal spatial unit to study spatial signatures, one which
meets the three requirements we outlined above.
%
They are indivisable in that a single EC will contain no delimiters, at most a
single anchor, and potentially none.
%
They are also internally consistent because they are delineated as the area
within the delimiters that contain at most one anchor.
%
And finally ET cells are exhaustive in that every location within the area of
interest is assigned to one and only one EC, providing full geographical
coverage without any overlap.
%
Due to space and focus constraints, we do not compare them empirically to
competing alternatives such as administrative units or uniform grids, but we
consider this endeavour a fruitful avenue for future research.


\subsection{Embedding form and function into spatial signatures}
\label{ssec:ss_ff}

%
This section covers the development of spatial signatures from a set of ET cells.
% - EC have to be described
ET cells take the role of the structural unit.
In themselves, they hold descriptive value in reflecting the configuration of
the urban environment. They also operate as a container, into which other
morphometric and functional characters can be embedded.
% - its character is used to grow Signatures
% - we aim to describe intrinsic character of each cell depending on itself and,
% importantly, its context
To "fill" these containers with more information,
we propose to collect a set of descriptors reflecting both form
and function so that, together, they capture an intertwined representation
of both dimensions.
%
In practice, this process will lead to a heterogeneous
mix of morphometric characters, capturing patterns of physical, built-up
environment; and functional characters, reflecting economic activity, amenities,
land use classification or historical importance.

\small
\begin{longtable}{p{5cm}p{4cm}p{4cm}l}
\caption{Excerpt of form characters used in the Barcelona case study. Implementation details are provided
in Jupyter notebooks available at \texttt{<anonymised for peer-review>}. The categorisation follows \cite{fleischmann2020measuring}.}
\label{tab:bcn_form_excerpt} \\
\toprule
                               index &                         element &                    context &     category \\
\midrule
\endfirsthead

\toprule
                               index &                         element &                    context &     category \\
\midrule
\endhead
\midrule
\multicolumn{4}{r}{{Continued on next page}} \\
\midrule
\endfoot

\bottomrule
\endlastfoot
\dots &                        \dots &                   \dots &    \dots \\
                                area &                        building &                   building &    dimension \\
                           perimeter &                        building &                   building &    dimension \\
                circular compactness &                        building &                   building &        shape \\
                          squareness &                        building &                   building &        shape \\
                   solar orientation &                        building &                   building & distribution \\
                    street alignment &                        building &                   building & distribution \\
                 coverage area ratio &               tessellation cell &          tessellation cell &    intensity \\
                            openness &                  street profile &             street segment & distribution \\
                              degree &                     street node &         neighbouring nodes & distribution \\
                  shared walls ratio &             adjacent buildings  &        adjacent buildings  & distribution \\
                                area &                       enclosure &                  enclosure &    dimension \\
                    local meshedness &                  street network &              nodes 5 steps & connectivity \\
          local closeness centrality &                  street network &              nodes 5 steps & connectivity \\
               perimeter wall length &             adjacent buildings  &           joined buildings &    dimension \\
               \dots &                        \dots &                   \dots &    \dots \\
\end{longtable}


 
Which exact characters to compile for a particular implementation of spatial signatures
will depend on the availability of data in that context.
Since this section outlines the process conceptually, we do not consider including any specific
list as useful as providing broad guidance on the kind of characters that should 
be aimed for when designing an application of the spatial signatures.
Any selection in this respect should aspire to reflect the nature of form and function in the
area of interest in as exhaustive a way as possible.
% Form/function example
As an example,
Table \ref{tab:bcn_form_excerpt} (\ref{tab:bcn_fn_excerpt}) contains an excerpt of Table
\ref{tab:form_bcn} (\ref{tab:fn_bcn}) in the supplementary
material, which captures all of the form (function) characters we use in the Barcelona illustration of
Section \ref{sec:app}.
%
We recommend building on the principles explored by \cite{dibble2019origin} and
\cite{fleischmann2021methodological}, and following the rules originally proposed by
\cite{sneath1973numerical}. These can be broadly summarised as
\emph{include as many characters present in literature as is feasible, while minimising
potential collinearity and limiting redundancy of information}. This guidance includes all
categories of form characters identified by \cite{fleischmann2020measuring} (ie. dimension,
shape, spatial distribution, intensity, connectivity, diversity) and as wide as possible of a range
for functional characters available for a given case study, including land cover/use,
employment/economic activity, and amenities. 
%
While the optimal ratio of form to function characters is to be aimed at balance, this may not
always be possible. Form characters can be derived from a small set of data sources (eg.
street networks and building footprints) while describing function relies on a larger set
of data. We do not see this as a limitation of the spatial signatures as much as one
of data availability. If anything, the joint approach of form and function encouraged
by our proposal ameliorates the problem given the interrelations between form and function
described above and that function has been found to be implicitly present in the description of form
(\citealp{caniggia2001architectural} in \citealp{kropf2009aspects}).

\begin{longtable}{p{5cm}p{3cm}p{5cm}}
\caption{Excerpt of function characters and transfer methods used in the Barcelona case study.
Implementation details are provided
in Jupyter notebooks available at \texttt{<anonymised for peer-review>}.}
\label{tab:bcn_fn_excerpt} \\
\toprule
                                        character & input spatial unit &                                    transfer method \\
\midrule
\endfirsthead

\toprule
                                        character & input spatial unit &                                    transfer method \\
\midrule
\endhead
\midrule
\multicolumn{3}{r}{{Continued on next page}} \\
\midrule
\endfoot

\bottomrule
\endlastfoot
\dots &                        \dots &                   \dots  \\

                                        population &              block &                  Building-based Dasymetric mapping \\
    number of other items that are not premises &              block &                                 Dasymetric mapping \\
                                        land use &             parcel &                            Spatial join (centroid) \\
                            number of dwellings &           building &                                     Attribute join \\
                                            parks &             points & Accessibility  - distance to nearest / \# within 15min \\
                                    restaurants &              point & Accessibility  - distance to nearest / \# within 15min \\
                                            trees &             points &                               Spatial join (count) \\
                                            NDVI &          raster 1m &                                        Zonal stats \\
                                            \dots &                        \dots &                   \dots  \\

\end{longtable}
\normalsize

                %%% Context %%%
\dani{Martin add paragraph on how we use context and why here}
In this step our proposal is to describe both the
intrinsic traits of each cell, but to also incorporate features from the immediate
spatial context.
%
It is to be noted that
every piece of information is considered within its spatial context.
%----------------------------------------------------------------------------------

% - interpolation
% - joins
Collecting characters at the ET cell level is only half the task to develop
spatial signatures. Given the granularity and multi-dimensionality of the
information at this stage, we need to combine it in a way that retains its
core characteristics but is easier to parse through.
%
We propose a feasible aggregation of ET cells into
spatial signatures using unsupervised learning. Again, it is not the role of
this section to single out a technique, since many exist including K-Means,
gaussian mixture models, or self-organizing maps \citep{kohonen1990self}, to
name a few. We note there is no need to impose a spatial contiguity constraint
as spatially contiguous clusters of cells in the same signature will emerge
thanks to the inherent spatial autocorrelation of data derived from mutually
overlapping \textit{contexts}.
%
These continuous groups of cells grouped in the same cluster is what we call
instances of a spatial signature.
% - each EC is characterizes by form and function of its immediate surroundings,
% allowing a feasible aggregation of ECs to SS.


% Section 4 - Empirics
\section{Illustration}
\label{sec:app}

% when talking about BCN mention
% https://www.researchgate.net/publication/254457523_Urban_form_and_compactness_of_morphological_homogeneous_districts_in_Barcelona_towards_an_automatic_classification_of_similar_built-up_structures_in_the_city#

% all technical details in to an appendix
% tables from excel should go to notebooks

%%%%%%%%%%%% Structure
% 4.1 (squeeze it in a page)
% introduction of cases - 1 per continent; geographical variation, take
% different cities, cultures, historical moments

%% method - top level outline of the method. Data, Form + convolution, Function, Clustering
% One para on how to build the data - F+F+Convolution, how it varies across
% examples (link to App.)

% Second para on how we generate signatures (clustering); once we have those we
% dissolve.

% 4.2 (a page and a half)
% Results; tell some stories to get reader on board

% how to read the map in an applied way

% things which are shared/consistent

% interesting stories from cases (BCN village centers, DeS slums)

% Section 5 - Conclusion
\section{Conclusions}
\label{sec:conclusions}

As a characterisation of space designed to understand urban environments, spatial
signatures have the potential to provide unique insight into the ways human populations
create and inhabit their cities. The illustrations shown in the the previous section
indicate that the resulting patchworks of signatures are consequences of the combination
of a broad range of aspects stretching from topography, through historical development
to the current use of space, each influencing the nature of the urban environment in its
specific way.

The variability is also reflected in the changing scale and shape of the environment,
with very granular and organic patterns in medieval development, rigid grids of
industrial era and vast unbuilt natural areas limiting the expansion of nearby cities.
In this context, the enclosed tessellation shows a high degree of the adaptability of
the spatial unit, based on the geographical position in which cells are generated and
consistently reflects the pattern of both built and unbuilt areas.

The combination of form and function within a single classification method can tell more
than either of them would be able to do alone. Spatial Signatures therefore bridge
purely morphological approaches based on concepts like morphological region
\citep{oliveira2020}, Local Climate Zone \citep{stewart2012} or Urban Structural Type
\citep{lehner2019} and functional classification as land use and land cover
\citep{georganos2018very} or mobility and population \citep{gale2016creating}, while
building on their abilities to comprehensively describe a single aspect of the
environment. That said, compared to the most commonly available land cover
classification, Spatial Signatures are complimentary as they turn the focus to urban
landscapes. Where former offer rich sets of classes in non-urban environment but only a
handful in cities, the latter tends to provide more detail in cities and less outside of
them, underyling the focus of Spatial Signatures as a classification designed to
understand urban environments.

The proposed method provides a step forward in understanding the environment in a
detailed, granular manner. Spatial Signatures can be seen as a commons space between
different fields exploring cities. Many of the disciplines understands at least one of
the components built into signatures, from morpohologists being familiar with
morphometric assessment, through demographers and population data to economists and
retail allocation. This overlap between the fields can facilitate interdisciplinary
research on top of singatures as a shared basis.

Spatial signatures can become a backbone for actions aiming under the umbrella of
Sustainable Development Goals, one which is data-informed and can be targeted due to its
inherent granularity. Furthermore, policy within the context of rapid urbanisation on
the global South can benefit from classification, which combines form and function as
development patterns and varying levels of their formality are not always aligned with
the functional aspects of such environments.

% D: this section needs a push on writing
Spatial signatures are the way of thinking rather than a fixed method using the exact
set of variables. It is a way of conceptualising built (and non-built) environment. The
outputs from different regions or countries are different manifestations of the same
idea, leading to conceptual consistency rather than a strictly numerical one. Specific
regions may have specific needs regarding the characterisation that need to be
acknowledged in the applied method. Furthermore, as illustrated in the previous section,
data environments vary significantly, and we cannot assume the same quality worldwide.
However, all such adaptations are still entirely within the conceptual notion of spatial
signature.

% from here to the end: D: this needs more careful writing, it reads more like a
% drafting of ideas
Further research should look deeper into how signatures from different places and
periods relate to each other and how they differ, to understand the rules which are
common and those which are geographically specific. Mirroring Tobler's first law of
geography \citep{tobler1970computer}, similar cities tend to be located closer to each
other, indicating the potential of identification of underlying geographical patterns
within signatures.

Spatial signatures as a way of thinking about space from the perspectives of form and
function form a link between currently fragmented research. The resulting
classification embodies a plurality of perspectives leading to a richer picture of the
urban environment and its better understanding.


\clearpage




\bibliographystyle{apalike}
\bibliography{refs}

\clearpage

\appendix
\section{Technical appendix}
\label{sec:appendix}

\subsection{The complete lists of used characters relfecting both form and function across case studies}
\subsubsection{Barcelona}

The data for the Barcelona case study has been retrieved from the Barcelona's City Hall
Open Data Service Open Data BCN available at
\href{https://opendata-ajuntament.barcelona.cat}{opendata-ajuntament.barcelona.cat},
\href{https://osm.org}{OpenStreetMap}, and Spanish Cadastre available at
\href{https://catastro.minhap.es}{catastro.minhap.es}. The data represents versions
available on December 01, 2020.

\small
\begin{longtable}{p{5cm}p{4cm}p{4cm}l}
\caption{The complete list of form characters used in the Barcelona case study. The implementation details are available
in Jupyter notebooks available at <anonymised for peer-review>.
Characters are adapted to enclosed tessellation from \cite{fleischmann2021methodological}.}
\label{tab:form_bcn} \\
\toprule
                                   index &                         element &                    context &     category \\
\midrule
\endfirsthead

\toprule
                               index &                         element &                    context &     category \\
\midrule
\endhead
\midrule
\multicolumn{4}{r}{{Continued on next page}} \\
\midrule
\endfoot

\bottomrule
\endlastfoot
                                area &                        building &                   building &    dimension \\
                           perimeter &                        building &                   building &    dimension \\
                      courtyard area &                        building &                   building &    dimension \\
                circular compactness &                        building &                   building &        shape \\
                             corners &                        building &                   building &        shape \\
                          squareness &                        building &                   building &        shape \\
        equivalent rectangular index &                        building &                   building &        shape \\
                          elongation &                        building &                   building &        shape \\
centroid - corner distance deviation &                        building &                   building &        shape \\
     centroid - corner mean distance &                        building &                   building &    dimension \\
                   solar orientation &                        building &                   building & distribution \\
                    street alignment &                        building &                   building & distribution \\
                      cell alignment &                        building &                   building & distribution \\
                 longest axis length &               tessellation cell &          tessellation cell &    dimension \\
                                area &               tessellation cell &          tessellation cell &    dimension \\
                circular compactness &               tessellation cell &          tessellation cell &        shape \\
        equivalent rectangular index &               tessellation cell &          tessellation cell &        shape \\
                   solar orientation &               tessellation cell &          tessellation cell & distribution \\
                 coverage area ratio &               tessellation cell &          tessellation cell &    intensity \\
                    street alignment &               tessellation cell &          tessellation cell & distribution \\
                              length &                  street segment &             street segment &    dimension \\
                               width &                  street profile &             street segment &    dimension \\
                            openness &                  street profile &             street segment & distribution \\
                     width deviation &                  street profile &             street segment &    diversity \\
                           linearity &                  street segment &             street segment &        shape \\
                        area covered &                  street segment &             street segment &    dimension \\
                 buildings per meter &                  street segment &             street segment &    intensity \\
                        area covered &                     street node &                street node &    dimension \\
                           alignment &          neighbouring buildings & neighbouring cells (queen) & distribution \\
                       mean distance &          neighbouring buildings & neighbouring cells (queen) & distribution \\
                 weighted neighbours &               tessellation cell & neighbouring cells (queen) & distribution \\
                        area covered &              neighbouring cells & neighbouring cells (queen) &    dimension \\
                        reached area &           neighbouring segments &      neighbouring segments &    dimension \\
                       reached cells &           neighbouring segments &      neighbouring segments &    intensity \\
                              degree &                     street node &         neighbouring nodes & distribution \\
 mean distance to neighbouring nodes &                     street node &         neighbouring nodes &    dimension \\
                  shared walls ratio &             adjacent buildings  &        adjacent buildings  & distribution \\
        mean inter-building distance &          neighbouring buildings &    cell queen neighbours 3 & distribution \\
         weighted reached enclosures & neighbouring tessellation cells &    cell queen neighbours 3 &    intensity \\
                                area &                       enclosure &                  enclosure &    dimension \\
                           perimeter &                       enclosure &                  enclosure &    dimension \\
                circular compactness &                       enclosure &                  enclosure &        shape \\
        equivalent rectangular index &                       enclosure &                  enclosure &        shape \\
           compactness-weighted axis &                       enclosure &                  enclosure &        shape \\
                   solar orientation &                       enclosure &                  enclosure & distribution \\
                 weighted neighbours &                       enclosure &                  enclosure & distribution \\
                      weighted cells &                       enclosure &                  enclosure &    intensity \\
                    local meshedness &                  street network &              nodes 5 steps & connectivity \\
                 mean segment length &                  street network &            segment 3 steps &    dimension \\
                   cul-de-sac length &                  street network &              nodes 3 steps &    dimension \\
                        node density &                  street network &              nodes 5 steps &    intensity \\
           proportion of cul-de-sacs &                  street network &              nodes 5 steps & connectivity \\
   proportion of 3-way intersections &                  street network &              nodes 5 steps & connectivity \\
   proportion of 4-way intersections &                  street network &              nodes 5 steps & connectivity \\
        degree weighted node density &                  street network &              nodes 5 steps &    intensity \\
          local closeness centrality &                  street network &              nodes 5 steps & connectivity \\
               perimeter wall length &             adjacent buildings  &           joined buildings &    dimension \\
                number of courtyards &             adjacent buildings  &           joined buildings &    intensity \\
                   square clustering &                  street network &             street network & connectivity \\
\end{longtable}


\begin{longtable}{p{5cm}p{3cm}p{5cm}}
\caption{The complete list of function characters and transfer methods used in the Barcelona case study. The implementation details are available
in Jupyter notebooks available at <anonymised for peer-review>.}
\label{tab:fn_bcn} \\
\toprule
                                         character & input spatial unit &                                    transfer method \\
\midrule
\endfirsthead

\toprule
                                         character & input spatial unit &                                    transfer method \\
\midrule
\endhead
\midrule
\multicolumn{3}{r}{{Continued on next page}} \\
\midrule
\endfoot

\bottomrule
\endlastfoot
                                        population &              block &                  Building-based Dasymetric mapping \\
                               number of car parks &              block &                                 Dasymetric mapping \\
       number of other items that are not premises &              block &                                 Dasymetric mapping \\
                                          land use &             parcel &                            Spatial join (centroid) \\
                               number of dwellings &           building &                                     Attribute join \\
                                       current use &           building &                                     Attribute join \\
                                               age &           building &                                     Attribute join \\
                                          heritage &             points &                     Accessibility  -\# within 15min \\
                                          heritage &           polygons &                                       Spatial join \\
   culture (cinemas, museums, libraries, theaters) &             points & Accessibility  - distance to nearest / \# within 15min \\
                                             parks &             points & Accessibility  - distance to nearest / \# within 15min \\
                                   economic census &            poitnts & Accessibility  - distance to nearest / \# within 15min \\
                                       restaurants &              point & Accessibility  - distance to nearest / \# within 15min \\
                                             trees &             points &                               Spatial join (count) \\
                                              NDVI &          raster 1m &                                        Zonal stats \\
\end{longtable}

\subsubsection{Medellin}

The data for the Medellin case study has been retrieved from the GeoMedellin Open Data
 portal available at
 \href{https://www.medellin.gov.co/geomedellin/}{medellin.gov.co/geomedellin/},
 \href{https://osm.org}{OpenStreetMap}, WorldPop gridded population estimates
 \citep{bondarenko2020census}, and a Sentinel 2 cloud-free composite
 \citep{CORBANE2020105737}. The data represents versions available on January 05, 2021.

\begin{longtable}{p{5cm}p{4cm}p{4cm}l}
\caption{The complete list of form characters used in the Medellin case study. The implementation details are available
in Jupyter notebooks available at <anonymised for peer-review>.
Characters are adapted to enclosed tessellation from \cite{fleischmann2021methodological}.}
\label{tab:form_med} \\
\toprule
                                index &                         element &                    context &     category \\
\midrule
\endfirsthead

\toprule
                                index &                         element &                    context &     category \\
\midrule
\endhead
\midrule
\multicolumn{4}{r}{{Continued on next page}} \\
\midrule
\endfoot

\bottomrule
\endlastfoot
                                area &                        building &                   building &    dimension \\
                            perimeter &                        building &                   building &    dimension \\
                        courtyard area &                        building &                   building &    dimension \\
                circular compactness &                        building &                   building &        shape \\
                                corners &                        building &                   building &        shape \\
                            squareness &                        building &                   building &        shape \\
        equivalent rectangular index &                        building &                   building &        shape \\
                            elongation &                        building &                   building &        shape \\
centroid - corner distance deviation &                        building &                   building &        shape \\
        centroid - corner mean distance &                        building &                   building &    dimension \\
                    solar orientation &                        building &                   building & distribution \\
                    street alignment &                        building &                   building & distribution \\
                        cell alignment &                        building &                   building & distribution \\
                    longest axis length &               tessellation cell &          tessellation cell &    dimension \\
                                area &               tessellation cell &          tessellation cell &    dimension \\
                circular compactness &               tessellation cell &          tessellation cell &        shape \\
        equivalent rectangular index &               tessellation cell &          tessellation cell &        shape \\
                    solar orientation &               tessellation cell &          tessellation cell & distribution \\
                    coverage area ratio &               tessellation cell &          tessellation cell &    intensity \\
                    street alignment &               tessellation cell &          tessellation cell & distribution \\
                                length &                  street segment &             street segment &    dimension \\
                                width &                  street profile &             street segment &    dimension \\
                            openness &                  street profile &             street segment & distribution \\
                        width deviation &                  street profile &             street segment &    diversity \\
                            linearity &                  street segment &             street segment &        shape \\
                        area covered &                  street segment &             street segment &    dimension \\
                    buildings per meter &                  street segment &             street segment &    intensity \\
                        area covered &                     street node &                street node &    dimension \\
                            alignment &          neighbouring buildings & neighbouring cells (queen) & distribution \\
                        mean distance &          neighbouring buildings & neighbouring cells (queen) & distribution \\
                    weighted neighbours &               tessellation cell & neighbouring cells (queen) & distribution \\
                        area covered &              neighbouring cells & neighbouring cells (queen) &    dimension \\
                        reached area &           neighbouring segments &      neighbouring segments &    dimension \\
                        reached cells &           neighbouring segments &      neighbouring segments &    intensity \\
                                degree &                     street node &         neighbouring nodes & distribution \\
    mean distance to neighbouring nodes &                     street node &         neighbouring nodes &    dimension \\
                    shared walls ratio &             adjacent buildings  &        adjacent buildings  & distribution \\
        mean inter-building distance &          neighbouring buildings &    cell queen neighbours 3 & distribution \\
            weighted reached enclosures & neighbouring tessellation cells &    cell queen neighbours 3 &    intensity \\
                                area &                       enclosure &                  enclosure &    dimension \\
                            perimeter &                       enclosure &                  enclosure &    dimension \\
                circular compactness &                       enclosure &                  enclosure &        shape \\
        equivalent rectangular index &                       enclosure &                  enclosure &        shape \\
            compactness-weighted axis &                       enclosure &                  enclosure &        shape \\
                    solar orientation &                       enclosure &                  enclosure & distribution \\
                    weighted neighbours &                       enclosure &                  enclosure & distribution \\
                        weighted cells &                       enclosure &                  enclosure &    intensity \\
                    local meshedness &                  street network &              nodes 5 steps & connectivity \\
                    mean segment length &                  street network &            segment 3 steps &    dimension \\
                    cul-de-sac length &                  street network &              nodes 3 steps &    dimension \\
                        node density &                  street network &              nodes 5 steps &    intensity \\
            proportion of cul-de-sacs &                  street network &              nodes 5 steps & connectivity \\
    proportion of 3-way intersections &                  street network &              nodes 5 steps & connectivity \\
    proportion of 4-way intersections &                  street network &              nodes 5 steps & connectivity \\
        degree weighted node density &                  street network &              nodes 5 steps &    intensity \\
            local closeness centrality &                  street network &              nodes 5 steps & connectivity \\
                perimeter wall length &             adjacent buildings  &           joined buildings &    dimension \\
                number of courtyards &             adjacent buildings  &           joined buildings &    intensity \\
                    square clustering &                  street network &             street network & connectivity \\
\end{longtable}


\begin{longtable}{p{5cm}p{3cm}p{5cm}}
\caption{The complete list of function characters and transfer methods used in the Medellin case study. The implementation details are available
in Jupyter notebooks available at <anonymised for peer-review>.}
\label{tab:fn_med} \\
\toprule
                                            character & input spatial unit &                                    transfer method \\
\midrule
\endfirsthead

\toprule
                                            character & input spatial unit &                                    transfer method \\
\midrule
\endhead
\midrule
\multicolumn{3}{r}{{Continued on next page}} \\
\midrule
\endfoot

\bottomrule
\endlastfoot
                trees &   points &                               Spatial join (count) \\
                parks & polygons &                                Distance to nearest \\
heritage large areas & polygons &                               Spatial join (boolean) \\
            land use & polygons &                                             tobler \\
                pois &   points & Accessibility  - distance to nearest / \# within 15min \\
        public spaces & polygons &             Accessibility  -  area within radius \\
            population &   raster &                                        Zonal stats \\
                NDVI &   raster &                                        Zonal stats \\
\end{longtable}

\subsubsection{Dar es Salaam}

The data for the Dar es Salaam case study has been retrieved from
\href{https://osm.org}{OpenStreetMap}, WorldPop gridded population estimates
\citep{bondarenko2020census}, a Copernicus Global Land Cover
\citep{marcel_buchhorn_2020_3939050}, VIIRS Night lights data (June 2020) \citep{elvidge2013viirs},
and a Sentinel 2 cloud-free composite \citep{CORBANE2020105737}. The data represents
versions available on December 18, 2020.

\begin{longtable}{p{5cm}p{4cm}p{4cm}l}
    \caption{The complete list of form characters used in the Dar es Salaam case study. The implementation details are available
    in Jupyter notebooks available at <anonymised for peer-review>.
    Characters are adapted to enclosed tessellation from \cite{fleischmann2021methodological}.}
    \label{tab:form_des} \\
    \toprule
                                   index &                         element &                    context &     category \\
    \midrule
    \endfirsthead

    \toprule
                                   index &                         element &                    context &     category \\
    \midrule
    \endhead
    \midrule
    \multicolumn{4}{r}{{Continued on next page}} \\
    \midrule
    \endfoot

    \bottomrule
    \endlastfoot
                                    area &                        building &                   building &    dimension \\
                               perimeter &                        building &                   building &    dimension \\
                          courtyard area &                        building &                   building &    dimension \\
                    circular compactness &                        building &                   building &        shape \\
                                 corners &                        building &                   building &        shape \\
                              squareness &                        building &                   building &        shape \\
            equivalent rectangular index &                        building &                   building &        shape \\
                              elongation &                        building &                   building &        shape \\
    centroid - corner distance deviation &                        building &                   building &        shape \\
         centroid - corner mean distance &                        building &                   building &    dimension \\
                       solar orientation &                        building &                   building & distribution \\
                        street alignment &                        building &                   building & distribution \\
                          cell alignment &                        building &                   building & distribution \\
                     longest axis length &               tessellation cell &          tessellation cell &    dimension \\
                                    area &               tessellation cell &          tessellation cell &    dimension \\
                    circular compactness &               tessellation cell &          tessellation cell &        shape \\
            equivalent rectangular index &               tessellation cell &          tessellation cell &        shape \\
                       solar orientation &               tessellation cell &          tessellation cell & distribution \\
                     coverage area ratio &               tessellation cell &          tessellation cell &    intensity \\
                                  length &                  street segment &             street segment &    dimension \\
                                   width &                  street profile &             street segment &    dimension \\
                                openness &                  street profile &             street segment & distribution \\
                         width deviation &                  street profile &             street segment &    diversity \\
                               linearity &                  street segment &             street segment &        shape \\
                            area covered &                  street segment &             street segment &    dimension \\
                     buildings per meter &                  street segment &             street segment &    intensity \\
                            area covered &                     street node &                street node &    dimension \\
                               alignment &          neighbouring buildings & neighbouring cells (queen) & distribution \\
                           mean distance &          neighbouring buildings & neighbouring cells (queen) & distribution \\
                     weighted neighbours &               tessellation cell & neighbouring cells (queen) & distribution \\
                            area covered &              neighbouring cells & neighbouring cells (queen) &    dimension \\
                           reached cells &           neighbouring segments &      neighbouring segments &    intensity \\
                            reached area &           neighbouring segments &      neighbouring segments &    dimension \\
                                  degree &                     street node &         neighbouring nodes & distribution \\
     mean distance to neighbouring nodes &                     street node &         neighbouring nodes &    dimension \\
            mean inter-building distance &          neighbouring buildings &    cell queen neighbours 3 & distribution \\
             weighted reached enclosures & neighbouring tessellation cells &    cell queen neighbours 3 &    intensity \\
                       reached neighbors & neighbouring tessellation cells &    cell queen neighbours 3 &    intensity \\
                            reached area & neighbouring tessellation cells &    cell queen neighbours 3 &    dimension \\
                                    area &                       enclosure &                  enclosure &    dimension \\
                               perimeter &                       enclosure &                  enclosure &    dimension \\
                    circular compactness &                       enclosure &                  enclosure &        shape \\
            equivalent rectangular index &                       enclosure &                  enclosure &        shape \\
               compactness-weighted axis &                       enclosure &                  enclosure &        shape \\
                       solar orientation &                       enclosure &                  enclosure & distribution \\
                     weighted neighbours &                       enclosure &                  enclosure & distribution \\
                          weighted cells &                       enclosure &                  enclosure &    intensity \\
                        local meshedness &                  street network &              nodes 5 steps & connectivity \\
                     mean segment length &                  street network &            segment 3 steps &    dimension \\
                       cul-de-sac length &                  street network &              nodes 3 steps &    dimension \\
                            reached area &                  street network &            segment 3 steps &    dimension \\
                           reached cells &                  street network &            segment 3 steps &    intensity \\
                            node density &                  street network &              nodes 5 steps &    intensity \\
               proportion of cul-de-sacs &                  street network &              nodes 5 steps & connectivity \\
       proportion of 3-way intersections &                  street network &              nodes 5 steps & connectivity \\
       proportion of 4-way intersections &                  street network &              nodes 5 steps & connectivity \\
            degree weighted node density &                  street network &              nodes 5 steps &    intensity \\
              local closeness centrality &                  street network &              nodes 5 steps & connectivity \\
                       square clustering &                  street network &             street network & connectivity \\
\end{longtable}


\begin{longtable}{p{5cm}p{3cm}p{5cm}}
    \caption{The complete list of function characters and transfer methods used in the Dar es Salaam case study. The implementation details are available
    in Jupyter notebooks available at <anonymised for peer-review>.}
    \label{tab:fn_des} \\
    \toprule
                                             character & input spatial unit &                                    transfer method \\
    \midrule
    \endfirsthead

    \toprule
                                             character & input spatial unit &                                    transfer method \\
    \midrule
    \endhead
    \midrule
    \multicolumn{3}{r}{{Continued on next page}} \\
    \midrule
    \endfoot

    \bottomrule
    \endlastfoot
    population & raster 100m &     Zonal stats \\
          NDVI &  raster 10m &     Zonal stats \\
    land cover &      raster &     Zonal stats \\
  night lights &      raster &     Zonal stats \\
\end{longtable}

\subsubsection{Houston}

The data for the Houston case study has been retrieved from
\href{https://osm.org}{OpenStreetMap}, Microsoft Building Footprints available from
\href{https://www.microsoft.com/en-us/maps/building-footprints}{microsoft.com/en-us/maps/building-footprints},
WorldPop gridded population estimates
\citep{bondarenko2020census}, a Copernicus Global Land Cover
\citep{marcel_buchhorn_2020_3939050}, VIIRS Night lights data (November 2019) \citep{elvidge2013viirs}, a Sentinel 2 cloud-free composite \citep{CORBANE2020105737}, and
the Longitudinal Employer-Household Dynamics data from US Census 2011 available from
\href{https://lehd.ces.census.gov/data/}{lehd.ces.census.gov/data/}. The data represents
versions available on January 13, 2021.

\begin{longtable}{p{5cm}p{4cm}p{4cm}l}
    \caption{The complete list of form characters used in the Houston case study. The implementation details are available
    in Jupyter notebooks available at <anonymised for peer-review>.
    Characters are adapted to enclosed tessellation from \cite{fleischmann2021methodological}.}
    \label{tab:form_hou} \\
    \toprule
                                   index &                         element &                    context &     category \\
    \midrule
    \endfirsthead

    \toprule
                                   index &                         element &                    context &     category \\
    \midrule
    \endhead
    \midrule
    \multicolumn{4}{r}{{Continued on next page}} \\
    \midrule
    \endfoot

    \bottomrule
    \endlastfoot
                                    area &                        building &                   building &    dimension \\
                               perimeter &                        building &                   building &    dimension \\
                          courtyard area &                        building &                   building &    dimension \\
                    circular compactness &                        building &                   building &        shape \\
                                 corners &                        building &                   building &        shape \\
                              squareness &                        building &                   building &        shape \\
            equivalent rectangular index &                        building &                   building &        shape \\
                              elongation &                        building &                   building &        shape \\
    centroid - corner distance deviation &                        building &                   building &        shape \\
         centroid - corner mean distance &                        building &                   building &    dimension \\
                       solar orientation &                        building &                   building & distribution \\
                        street alignment &                        building &                   building & distribution \\
                          cell alignment &                        building &                   building & distribution \\
                     longest axis length &               tessellation cell &          tessellation cell &    dimension \\
                                    area &               tessellation cell &          tessellation cell &    dimension \\
                    circular compactness &               tessellation cell &          tessellation cell &        shape \\
            equivalent rectangular index &               tessellation cell &          tessellation cell &        shape \\
                       solar orientation &               tessellation cell &          tessellation cell & distribution \\
                     coverage area ratio &               tessellation cell &          tessellation cell &    intensity \\
                                  length &                  street segment &             street segment &    dimension \\
                                   width &                  street profile &             street segment &    dimension \\
                                openness &                  street profile &             street segment & distribution \\
                         width deviation &                  street profile &             street segment &    diversity \\
                               linearity &                  street segment &             street segment &        shape \\
                            area covered &                  street segment &             street segment &    dimension \\
                     buildings per meter &                  street segment &             street segment &    intensity \\
                               alignment &          neighbouring buildings & neighbouring cells (queen) & distribution \\
                           mean distance &          neighbouring buildings & neighbouring cells (queen) & distribution \\
                     weighted neighbours &               tessellation cell & neighbouring cells (queen) & distribution \\
                            area covered &              neighbouring cells & neighbouring cells (queen) &    dimension \\
                           reached cells &           neighbouring segments &      neighbouring segments &    intensity \\
                            reached area &           neighbouring segments &      neighbouring segments &    dimension \\
                                  degree &                     street node &         neighbouring nodes & distribution \\
     mean distance to neighbouring nodes &                     street node &         neighbouring nodes &    dimension \\
            mean inter-building distance &          neighbouring buildings &    cell queen neighbours 3 & distribution \\
             weighted reached enclosures & neighbouring tessellation cells &    cell queen neighbours 3 &    intensity \\
                       reached neighbors & neighbouring tessellation cells &    cell queen neighbours 3 &    intensity \\
                            reached area & neighbouring tessellation cells &    cell queen neighbours 3 &    dimension \\
                                    area &                       enclosure &                  enclosure &    dimension \\
                               perimeter &                       enclosure &                  enclosure &    dimension \\
                    circular compactness &                       enclosure &                  enclosure &        shape \\
            equivalent rectangular index &                       enclosure &                  enclosure &        shape \\
               compactness-weighted axis &                       enclosure &                  enclosure &        shape \\
                       solar orientation &                       enclosure &                  enclosure & distribution \\
                     weighted neighbours &                       enclosure &                  enclosure & distribution \\
                          weighted cells &                       enclosure &                  enclosure &    intensity \\
                        local meshedness &                  street network &              nodes 5 steps & connectivity \\
                     mean segment length &                  street network &            segment 3 steps &    dimension \\
                       cul-de-sac length &                  street network &              nodes 3 steps &    dimension \\
                            reached area &                  street network &            segment 3 steps &    dimension \\
                           reached cells &                  street network &            segment 3 steps &    intensity \\
                            node density &                  street network &              nodes 5 steps &    intensity \\
               proportion of cul-de-sacs &                  street network &              nodes 5 steps & connectivity \\
       proportion of 3-way intersections &                  street network &              nodes 5 steps & connectivity \\
       proportion of 4-way intersections &                  street network &              nodes 5 steps & connectivity \\
            degree weighted node density &                  street network &              nodes 5 steps &    intensity \\
              local closeness centrality &                  street network &              nodes 5 steps & connectivity \\
                       square clustering &                  street network &             street network & connectivity \\
\end{longtable}

\begin{longtable}{p{5cm}p{3cm}p{5cm}}
    \caption{The complete list of function characters and transfer methods used in the Houston case study. The implementation details are available
    in Jupyter notebooks available at <anonymised for peer-review>.}
    \label{tab:fn_hou} \\
    \toprule
                                             character & input spatial unit &                                    transfer method \\
    \midrule
    \endfirsthead

    \toprule
                                             character & input spatial unit &                                    transfer method \\
    \midrule
    \endhead
    \midrule
    \multicolumn{3}{r}{{Continued on next page}} \\
    \midrule
    \endfoot

    \bottomrule
    \endlastfoot
    population &  raster 100m &              Zonal stats \\
          NDVI &   raster 10m &              Zonal stats \\
    land cover &       raster &              Zonal stats \\
  night lights &       raster &              Zonal stats \\
    employment & census block & Dasymetric interpolation \\
historical sites &        point &            Accessibility \\
\end{longtable}

\subsubsection{Singapore}
The data for the Singapore case study has been retrieved from
\href{https://osm.org}{OpenStreetMap}, Singapore Open data portal available at
\href{https://data.gov.sg/}{data.gov.sg}, WorldPop gridded population estimates
\citep{bondarenko2020census}, VIIRS Night lights data (November 2019) \citep{elvidge2013viirs},
and a Sentinel 2 cloud-free composite \citep{CORBANE2020105737}. The data represents
versions available on February 1, 2021.


\begin{longtable}{p{5cm}p{4cm}p{4cm}l}
    \caption{The complete list of form characters used in the Singapore case study. The implementation details are available
    in Jupyter notebooks available at <anonymised for peer-review>.
    Characters are adapted to enclosed tessellation from \cite{fleischmann2021methodological}.}
    \label{tab:form_sin} \\
    \toprule
                                   index &                         element &                    context &     category \\
    \midrule
    \endfirsthead

    \toprule
                                   index &                         element &                    context &     category \\
    \midrule
    \endhead
    \midrule
    \multicolumn{4}{r}{{Continued on next page}} \\
    \midrule
    \endfoot

    \bottomrule
    \endlastfoot
                                    area &                        building &                   building &    dimension \\
                               perimeter &                        building &                   building &    dimension \\
                          courtyard area &                        building &                   building &    dimension \\
                    circular compactness &                        building &                   building &        shape \\
                                 corners &                        building &                   building &        shape \\
                              squareness &                        building &                   building &        shape \\
            equivalent rectangular index &                        building &                   building &        shape \\
                              elongation &                        building &                   building &        shape \\
    centroid - corner distance deviation &                        building &                   building &        shape \\
         centroid - corner mean distance &                        building &                   building &    dimension \\
                       solar orientation &                        building &                   building & distribution \\
                        street alignment &                        building &                   building & distribution \\
                          cell alignment &                        building &                   building & distribution \\
                     longest axis length &               tessellation cell &          tessellation cell &    dimension \\
                                    area &               tessellation cell &          tessellation cell &    dimension \\
                    circular compactness &               tessellation cell &          tessellation cell &        shape \\
            equivalent rectangular index &               tessellation cell &          tessellation cell &        shape \\
                       solar orientation &               tessellation cell &          tessellation cell & distribution \\
                     coverage area ratio &               tessellation cell &          tessellation cell &    intensity \\
                        street alignment &               tessellation cell &          tessellation cell & distribution \\
                                  length &                  street segment &             street segment &    dimension \\
                                   width &                  street profile &             street segment &    dimension \\
                                openness &                  street profile &             street segment & distribution \\
                         width deviation &                  street profile &             street segment &    diversity \\
                               linearity &                  street segment &             street segment &        shape \\
                            area covered &                  street segment &             street segment &    dimension \\
                     buildings per meter &                  street segment &             street segment &    intensity \\
                            area covered &                     street node &                street node &    dimension \\
                               alignment &          neighbouring buildings & neighbouring cells (queen) & distribution \\
                           mean distance &          neighbouring buildings & neighbouring cells (queen) & distribution \\
                     weighted neighbours &               tessellation cell & neighbouring cells (queen) & distribution \\
                            area covered &              neighbouring cells & neighbouring cells (queen) &    dimension \\
                            reached area &           neighbouring segments &      neighbouring segments &    dimension \\
                           reached cells &           neighbouring segments &      neighbouring segments &    intensity \\
                                  degree &                     street node &         neighbouring nodes & distribution \\
     mean distance to neighbouring nodes &                     street node &         neighbouring nodes &    dimension \\
                      shared walls ratio &             adjacent buildings  &        adjacent buildings  & distribution \\
            mean inter-building distance &          neighbouring buildings &    cell queen neighbours 3 & distribution \\
             weighted reached enclosures & neighbouring tessellation cells &    cell queen neighbours 3 &    intensity \\
                                    area &                       enclosure &                  enclosure &    dimension \\
                               perimeter &                       enclosure &                  enclosure &    dimension \\
                    circular compactness &                       enclosure &                  enclosure &        shape \\
            equivalent rectangular index &                       enclosure &                  enclosure &        shape \\
               compactness-weighted axis &                       enclosure &                  enclosure &        shape \\
                       solar orientation &                       enclosure &                  enclosure & distribution \\
                     weighted neighbours &                       enclosure &                  enclosure & distribution \\
                          weighted cells &                       enclosure &                  enclosure &    intensity \\
                        local meshedness &                  street network &              nodes 5 steps & connectivity \\
                     mean segment length &                  street network &            segment 3 steps &    dimension \\
                       cul-de-sac length &                  street network &              nodes 3 steps &    dimension \\
                            node density &                  street network &              nodes 5 steps &    intensity \\
               proportion of cul-de-sacs &                  street network &              nodes 5 steps & connectivity \\
       proportion of 3-way intersections &                  street network &              nodes 5 steps & connectivity \\
       proportion of 4-way intersections &                  street network &              nodes 5 steps & connectivity \\
            degree weighted node density &                  street network &              nodes 5 steps &    intensity \\
              local closeness centrality &                  street network &              nodes 5 steps & connectivity \\
                   perimeter wall length &             adjacent buildings  &           joined buildings &    dimension \\
                    number of courtyards &             adjacent buildings  &           joined buildings &    intensity \\
                       square clustering &                  street network &             street network & connectivity \\
    \end{longtable}

\begin{longtable}{p{5cm}p{3cm}p{5cm}}
    \caption{The complete list of function characters and transfer methods used in the Singapore case study. The implementation details are available
    in Jupyter notebooks available at <anonymised for peer-review>.}
    \label{tab:fn_sin} \\
    \toprule
                                                character & input spatial unit &                                    transfer method \\
    \midrule
    \endfirsthead

    \toprule
                                                character & input spatial unit &                                    transfer method \\
    \midrule
    \endhead
    \midrule
    \multicolumn{3}{r}{{Continued on next page}} \\
    \midrule
    \endfoot

    \bottomrule
    \endlastfoot
    NDVI &  raster &         Zonal stats \\
population &  raster &         Zonal stats \\
night lights &  raster &         Zonal stats \\
eating est &   point &       Accessibility \\
supermarkets &   point &       Accessibility \\
land use & polygon & Areal interpolation \\
   parks &   point &       Accessibility \\
monuments &   point &       Accessibility \\
\end{longtable}

\normalsize

\subsection{Clustergrams}

\begin{figure}
    \includegraphics[width=\linewidth]{figures/clustergram_bcn.png}
    \caption{Clustergram (truncated along the vertical axis) illustrating the behaviour
    of dataset in different clustering options. The result suggests two different solutions,
    a (very) conservative one with 4 classes and the other with 16 classes. For the purpose of
    spatial signatures, it is more suitable a solution with more classes providing more detailed
    classification. Therefore, we select 16 classes.
    }
    \label{fig:cgram_bcn}
\end{figure}

\begin{figure}
    \includegraphics[width=\linewidth]{figures/clustergram_med.png}
    \caption{Clustergram (truncated along the vertical axis) illustrating the behaviour
    of dataset in different clustering options. The result suggests 3 different
    solutions, a (very) conservative one with 4 classes, middle option with 11 classes
    and the detailed one with 19 classes. For the purpose of spatial signatures, it is
    more suitable a solution with more classes providing more detailed classification.
    Therefore, we select 19 classes.
    }
    \label{fig:cgram_med}
\end{figure}

\begin{figure}
    \includegraphics[width=\linewidth]{figures/clustergram_des.pdf}
    \caption{Clustergram (truncated along the vertical axis) illustrating the behaviour of dataset in different clustering options. The optimal number of clusters derived using the clustergram is 17.}
    \label{fig:cgram_des}
\end{figure}

\begin{figure}
    \includegraphics[width=\linewidth]{figures/clustergram_hou.png}
    \caption{Clustergram (truncated along the vertical axis) illustrating the behaviour
    of dataset in different clustering options. The result suggests 3 different
    solutions, a (very) conservative one with 3 classes, middle option with 6 classes
    and the detailed one with 10 classes. For the purpose of spatial signatures, it is
    more suitable a solution with more classes providing more detailed classification.
    Therefore, we select 10 classes.}
    \label{fig:cgram_hou}
\end{figure}

\begin{figure}
    \includegraphics[width=\linewidth]{figures/clustergram_sin.png}
    \caption{Clustergram (truncated along the vertical axis) illustrating the behaviour
    of dataset in different clustering options. The result suggests 3 different
    solutions, a (very) conservative one with 3 classes, middle option with 6 classes
    and the detailed one with 12 classes. For the purpose of spatial signatures, it is
    more suitable a solution with more classes providing more detailed classification.
    Therefore, we select 12 classes.}
    \label{fig:cgram_sin}
\end{figure}

\clearpage

\end{document}
