\subsection{Existing gaps}
\label{sec:lit_gaps}

% Union % The union between form and function
% - theory of the interconnectedness of the two
% - form <-> function

% - some papers combining both (from umm - Bourdic, Serra, partially Alexiou, some
%   resilience work; add other) -- check Angel's papers -- productivity/econ, --
%   efficiency -- environmental, fiscal (OECD) argument -- social integration and
%   isolation -- accessibility

% -- papers coming from morphological background (mostly) Bourdic, Salat and Nowacki
% 2012 - Land use, mobility, water, biodiversity, equity, economy, waste, culture,
% energy Serra, Psarra and O'Brien 2018 - Income deprivation (IMD) Song and Knaap 2007;
% Song, Popkin and Larsen 2013 - Land use, accessibility, transport Torrens 2008 -
% occupier profile Zheng et al 2014 - population density, proximity Ewing 2002 (SGA) -
% population (county level) Venerandi 2019 - social housing, commerce & services, job
% accessibility, housing stock Alexiou 2016 - proximity, very little form

% OECD 2018 - top level focus on sprawl, mostly population

% The link between the two (f+f) seems to be very weak and not many people combine both
% into a single classification method. Furthermore, if they do, the one or the other is
% just very simplified (population only, density only). Alternatively, they combine them
% reasonably (Bourdic 2012), but the result lacks granularity (their work is on city
% scale, Ewing on county scale).

Whilst literature often focuses either on form or on function, the two can be studies
independently but and deeply interconnected. Form reflects and influences function and
vice versa (REF). Therefore, any classification of built environment which aims to
provide a comprehensive picture of the reality needs to work with both aspects at the
same time. One example of a comprehensive combination of form and function into a
singular classification is the work of \cite{bourdic2012}, proposing an inclusive system
of spatial indicators ranging from form to biodiversity, culture and energy on a scale
of individual cities. Common are links between form and land use
\citep{song2007,song2013,bourdic2012}, where some authors even consider land use a part
of form characterisation \citep{dibble2019origin}. Global availability and
interpretability of available data \citep{pesaresi2019ghs,sorichetta2015} fuels research
linking form and population density
\citep{ahlfeldt2019,ewing2002measuring,zheng2014urban,oecd2018rethinking}, alongside
studies embedding accessibility and proximity to points of interests into their
frameworks \citep{alexiou2016a,venerandi2019machine}. However, the body of research
directly working with both form and function in a single framework is not large and a
balance between both aspects varies.


% Limits (gaps)
% - limits of existing methods -- where these methods can't deliver --- detail,
%   comprehensiveness, scalability (each lacks at least one) --- data requirements (some
%   are dependent on detailed data (Berghauser Pont, high-res RS)) ---- this could
%   conclude with limits of open RS data "forcing" us to work with morphometrics -> link
%   to the last part ---- we need training data before we can go RS way - we need theory
%   before
% - limits of function
% - limits of existing existing work combining two opportunities to cross-pollinate to
%   each other / hard for field to talk to each other SpSig should be unifying (link
%   back to this section from section 3)
From the perspective of built environment classification, the existing methods have
often limits, mostly related to detail, comprehensiveness and scalability, lacking at
least one of them. The situation is similar in classification based on form, function,
as well as their combination. Detail reflects spatial granularity of resulting
classification, where more granular, i.e. more detailed, unit has the ability to capture
smaller nuances of the urban environment and better reflect local characters or a place.
Methods based on a unit which can be further subdivided
\citep{dibble2019origin,jochem2020,araldi2019,gil2012}, therefore does not ensure
internal homogeneity, can result in classes driven by the heterogeneity instead of the
unit instead of the actual pattern of built environment. Comprehensiveness refers to the
number of characters (variables) used in the classification procedure. Small sets of
characters as in \cite{bobkova2019} or \cite{serra2018a} are prone to a selection bias
and will less likely reflect the complexity of the urban environment. Finally,
scalability reflects the ability of the proposed method to scale up to large extents of
metropolitan areas or national-level studies. While some works illustrate such a
potential \citep{jochem2020, schirmer2015,bobkova2019,araldi2019}, others which may
overcome other issues are less likely to scale from their original limits
\citep{dibble2019origin}. Furthermore, computational scalability can be limited by data
availability. Methods dependent on a high amount of detailed vector data
\citep{bobkova2019} or specific local demographic information (REF) can be hardly
applied in other contexts where such input is not available.
% TODO: Add references to function lit. The text is general and covers both now.


\citep{}
% --- touch on scale issue - What was this about? Is it covered above?



% more interaction between fields
