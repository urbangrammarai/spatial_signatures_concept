\subsection{Existing gaps}
\label{sec:lit_gaps}
% Limits (gaps)
% - limits of existing methods
% -- where these methods can't deliver
% --- detail, comprehensiveness, scalability (each lacks at least one)
% --- data requirements (some are dependent on detailed data (Berghauser Pont, high-res RS))
% ---- this could conclude with limits of open RS data "forcing" us to work with morphometrics -> link to the last part
% ---- we need training data before we can go RS way - we need theory before
% - limits of function
% - limits of existing existing work combining two
% opportunities to cross-pollinate to each other / hard for field to talk to each other
% SpSig should be unifying (link back to this section from section 3)
However, all the methods above have certain limits, mostly related to detail, comprehensiveness and scalability, lacking at least on them. Detail reflects spatial granularity of resulting classification, where more granular, i.e. more detailed, unit has the ability to capture smaller nuances of the urban environment and better reflect local characters or a place. Methods based on a unit which can be further subdivided \citep{dibble2019origin,jochem2020,araldi2019,gil2012}, therefore does not ensure internal homogeneity, can result in classes driven by the heterogeneity instead of the unit instead of the actual pattern of urban form. Comprehensiveness refers to the number of characters (variables) used in the classification procedure. Small sets of characters as in \cite{bobkova2019} or \cite{serra2018a} are prone to a selection bias and will less likely reflect the complexity of the urban environment. Finally, scalability reflects the ability of the proposed method to scale up to large extents of metropolitan areas or national-level studies. While some works illustrate such a potential \citep{jochem2020, schirmer2015,bobkova2019,araldi2019}, others which may overcome other issues are less likely to scale from their original limits \citep{dibble2019origin}. Furthermore, computational scalability can be limited by data availability. Methods dependent on a high amount of detailed vector data \citep{bobkova2019} can be hardly applied in other contexts where such input is not available.

% Union
%% The union between form and function
% - theory of the interconnectedness of the two
% - form <-> function

% - some papers combining both (from umm - Bourdic, Serra, partially Alexiou, some resilience work; add other)
% -- check Angel's papers
% -- productivity/econ, 
% -- efficiency
% -- environmental, fiscal (OECD) argument
% -- social integration and isolation
% -- accessibility

% -- papers coming from morphological background (mostly)
% Bourdic, Salat and Nowacki 2012 - Land use, mobility, water, biodiversity, equity, economy, waste, culture, energy
% Serra, Psarra and O'Brien 2018 - Income deprivation (IMD)
% Song and Knaap 2007; Song, Popkin and Larsen 2013 - Land use, accessibility, transport
% Torrens 2008 - occupier profile
% Zheng et al 2014 - population density, proximity
% Ewing 2002 (SGA) - population (county level)
% Venerandi 2019 - social housing, commerce & services, job accessibility, housing stock
% Alexiou 2016 - proximity, very little form

% OECD 2018 - top level focus on sprawl, mostly population

% The link between the two (f+f) seems to be very weak and not many people combine both into a single classification method. Furthemore, if they do, the one or the other is just very simplified (population only, density only). Alternatively, they combine them reasonably (Bourdic 2012), but the result lacks granularity (their work is on city scale, Ewing on county scale).

% --- touch on scale issue



