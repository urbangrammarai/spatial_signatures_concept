\subsection{Blending Form \& Function}
\label{sec:lit_gaps}

% The little there is on FF, much implicit
Whilst much of the literature focuses either form or function, the two
are deeply interconnected. Function develops in the context provided by
form; and, over time, form adapts and encodes function.
%
Nevertheless, there exist a few attempts to classify urban spaces considering
both form and function. \cite{bourdic2012} propose a comprehensive
classification based on indicators ranging from form to biodiversity, culture and energy on a scale of individual cities.
%
There are also a few studies considering the link between form and land use
\citep{song2007,song2013,bourdic2012}, with some authors even including land
use a component of form characterisations \citep{dibble2019origin}.
% Form and one function
Even when the two are combined, the scope included of either, particularly
function, is narrow.
%
The geodemographic tradition
\citep{harris2005,webber2018} studies populations based on where they live.
Although this considers both, the focus is very much on the residential
function, leaving aside others such as employment or amenities.
%
Recent years have seen work at the global scale connecting form and population
density \citep{ewing2002measuring,zheng2014urban,oecd2018rethinking}, many
facilitated by the appearance of new datasets (e.g.
\citealp{pesaresi2019ghs,sorichetta2015}), alongside studies embedding
accessibility and proximity to points of interests into their frameworks
\citep{alexiou2016a,venerandi2019machine}.
%
Nevertheless, the
body of research directly working with both form and function in a single
framework is not large and tends to focus on particular functions.

% Benefits of blended FF
% Gap for blended FF
%% FF are deeply interconnected, and some insights require the two
%% It is the combination of the two that encodes history, cultury, technology, etc.
%% From a pragmatic perspective, considering the two feeds more information, which leads to more robust representations

Therefore, any classification of
built environment which aims to provide a comprehensive picture of the reality
needs to work with both aspects at the same time. 


