\subsection{Blending Form \& Function}
\label{sec:lit_gaps}

% The little there is on FF, much implicit
Whilst much of the literature focuses either form or function, the two
are deeply interconnected. Function develops in the context provided by
form; and, over time, form adapts and encodes function.
%
However, there exists a few attempts to classify urban spaces considering
both jointly. \cite{bourdic2012} propose a comprehensive
classification based on indicators ranging from form to biodiversity, culture and energy on a scale of individual cities.
%
Several studies consider the link between form and land use
\citep{song2007,song2013,bourdic2012}, with some authors even including land
use a component of form characterisations \citep{dibble2019origin}.
% Form and one function
At any rate, even when the two are combined, the scope of either,
particularly function, is narrow rather than all-encompassing.
%
For example, the geodemographic tradition
\citep{harris2005,webber2018} studies populations based on where they live.
Although this considers both, the focus is very much on the residential
function, leaving aside others such as employment or amenities.
%
Recent years have also seen work at the global scale connecting form and
population density
\citep{ewing2002measuring,zheng2014urban,oecd2018rethinking}, many facilitated
by the appearance of new datasets (e.g.
\citealp{pesaresi2019ghs,sorichetta2015}), alongside studies embedding
accessibility and proximity to points of interests into their frameworks
\citep{alexiou2016a,venerandi2019machine}.
%
Nevertheless, the
body of research directly working with both form and function in a single
framework is limited and tends to focus on particular functions.


