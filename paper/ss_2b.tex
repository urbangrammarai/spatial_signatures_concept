% Function
\subsection{Function}
\label{sec:lit_function}
%% Function as a content
% - geodem literature and econ urban spatial structure about the structure of employment and econ activity
% - Dani? (mostly geodem, economics etc.)
% --- touch on scale issue


% Definition: function is what
Urban function considers environments based on the activities that take place
within them.
%
The focus is thus not on what a space ``looks
like'', as it is the case on urban form, but on ``what it is used for''; what
activities occur within cities, how they are spatially configured, and how
they relate to each other are key questions in this context.
%
% The study of function is much more scattered across different literatures.
To the extent cities compress space and time to concentrate human activity of
very diverse nature, the study of function is relevant to a variety of
fields and is undertaken by a wider constituency of researchers. Disciplines
as disparate as geography, economics or environmental sciences, to name only a
few, have contributed in their own way to our understanding of urban
function.
% Policy
Furthermore, because function has direct implications for a wide range of
social and environmental processes at different geographic scales, their study
also falls within the realm of policy.
%
Given the breadth of perspectives and goals, research on urban form is
difficult to classify and a complete overview of its contributions is beyond
the scope of this paper. Instead, here we will highlight what we consider the
most relevant domains involved in the study of urban form: GIScience,
environmental sciences, urban and public economics, urban and transport
geography, and sociology.

% Environmental and GI sciences: LU Vs LC
%% GISc/EnvSci --> Land use (Vs cover)
%% Ecology
%% XXX --> environmental sustainability (energy

Land-use Vs Land-cover (e.g. \citealp{fisher2005land, doi:10.1080/17474230802434187})

- Confusion + mostly focused on natural landscapes.
- Plenty of classifications (e.g. CORINE, global CC)

Lots of progress on inferring land use from RS, much of this from the Urban
Remote Sensing community \citep{weng2018urban}, thanks to object-based image
analysis advances (OBIA, e.g. \citealp{prasad2015remotely}); machine
learning (e.g. \citealp{kuffer2016slums, georganos2018very, JOCHEM2018104})
and, more recently, computer vision (e.g. \citealp{stark2020satellite, geiss2020deep}).

% Social Sciences
%% Urban economics --> density and agglomeration
%% Public economics --> public finance sustainability
%% Sociology --> social integration/segregation
%% Transport geography --> accessibility


