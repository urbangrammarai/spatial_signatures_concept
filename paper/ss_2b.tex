% Function
\subsection{Function}
\label{sec:lit_function}
%% Function as a content
% - geodem literature and econ urban spatial structure about the structure of employment and econ activity
% - Dani? (mostly geodem, economics etc.)
% --- touch on scale issue


% Union
%% The union between form and function
% - theory of the interconnectedness of the two
% - form <-> function

% - some papers combining both (from umm - Bourdic, Serra, partially Alexiou, some resilience work; add other)
% -- check Angel's papers
% -- productivity/econ, 
% -- efficiency
% -- environmental, fiscal (OECD) argument
% -- social integration and isolation
% -- accessibility

% -- papers coming from morphological background (mostly)
% Bourdic, Salat and Nowacki 2012 - Land use, mobility, water, biodiversity, equity, economy, waste, culture, energy
% Serra, Psarra and O'Brien 2018 - Income deprivation (IMD)
% Song and Knaap 2007; Song, Popkin and Larsen 2013 - Land use, accessibility, transport
% Torrens 2008 - occupier profile
% Zheng et al 2014 - population density, proximity
% Ewing 2002 (SGA) - population (county level)
% Venerandi 2019 - social housing, commerce & services, job accessibility, housing stock
% Alexiou 2016 - proximity, very little form

% OECD 2018 - top level focus on sprawl, mostly population

% The link between the two (f+f) seems to be very weak and not many people combine both into a single classification method. Furthemore, if they do, the one or the other is just very simplified (population only, density only). Alternatively, they combine them reasonably (Bourdic 2012), but the result lacks granularity (their work is on city scale, Ewing on county scale).

% --- touch on scale issue



