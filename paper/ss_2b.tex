% Function
\subsection{Function}
\label{sec:lit_function}
%% Function as a content
% - geodem literature and econ urban spatial structure about the structure of employment and econ activity
% - Dani? (mostly geodem, economics etc.)
% --- touch on scale issue


% Definition: function is what
Urban function considers environments based on the activities that take place
within them.
%
The focus is thus not on what a space ``looks
like'', as it is the case on urban form, but on ``what it is used for''. What
activities occur within cities, how they are spatially configured, and how
they relate to each other are key questions in this context.
%
% The study of function is much more scattered across different literatures.
To the extent cities compress space and time to concentrate human activity of
very diverse nature, the study of function is relevant to a variety of
fields and is undertaken by a wider constituency of researchers. Disciplines
as disparate as geography, economics or environmental sciences have contributed in their own way to our understanding of urban
function.
% Policy
Furthermore, because function has direct implications for a wide range of
social and environmental processes at different geographic scales, their study
also falls within the realm of policy.
%
Given the breadth of perspectives and goals, a complete overview of its
contributions is beyond
the scope of this paper. Instead, here we highlight what we consider the
most relevant domains involved: environmental
sciences, urban and public economics, urban and transport geography, planning, and
sociology.

% Environmental and GI sciences: LU Vs LC
Environmental sciences have long considered urban function
in the context of the broader interest on understanding the natural
characteristics of the surface of the Earth.
%% GISc/EnvSci --> Land use (Vs cover)
An area that has attracted much effort relates to the development of
classifications of land cover and land use, the former describing the nature of
surfaces while the latter focusing on how those surfaces are used. Several
land cover classifications are available (e.g. CORINE,
        \citealp{europeanenvironmentagency1990}, in Europe;
the National Land Cover Database, \citealp{homer2012national}, in the US; or
the Land Cover CCI, \citealp{defourny2012land}, globally), as well as some
for land use (e.g. the Urban Atlas project, \citealp{urban_atlas}).
% Urban Remote Sensing
While much of this research is not focused on urban environments, the urban
remote sensing community \citep{weng2018urban} is building a more explicit
bridge between these approaches and the study of cities
(e.g. \citealp{kuffer2016slums, georganos2018very, JOCHEM2018104, prasad2015remotely, stark2020satellite}).

A wide array of disciplines have developed more specific interests in urban
function.
%% Ecology
Sustainability studies, for example, are interested in how function is
configured within and across cities in so far as it relates to the level of
emissions \citep{angel2018shape} or energy consumption \citep{silva2017urban}.
%%% Check refs on OECD sprawl report
% Social Sciences
The social sciences have a long-standing interest on the spatial
configuration of form because it affects several outcomes of prime interest.
Depending on the nature of these outcomes, form is conceptualised in one or
another way.
%% Urban economics --> density and agglomeration
Urban economics pays special attention to density of economic activity and, by
extension, of population \citep{ahlfeldt2019, duranton2020economics}, since
density is intimately related to theories of agglomeration, one of the
intellectual pillars of the field.
%% Public economics --> public finance sustainability
Public economics has paid attention the
configuration of urban function to the extent that it determines the
efficiency of certain public services provided by local governments
\citep{carruthers2003urban, hortas2010does}.
%% Sociology --> social integration/segregation
Sociologists and planners have also found that different spatial configurations of function
over space is associated with different degrees of social mobility
\citep{ewing2016does} or socio-economic deprivation
\citep{venerandi2018scalable}.
%% Transport geography --> accessibility
More generally, transport researchers have built a robust body of knowledge
linking urban function and its spatial distribution to travel behaviour
\citep{boarnet2001travel}, sustainability \citep{sevtuk2020does}, or accessibility
to jobs \citep{horner2004exploring} and amenities \citep{diamond2013economics}, with clear implications for socio-economic disparities.


