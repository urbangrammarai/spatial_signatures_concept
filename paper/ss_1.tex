\section{Introduction}
\label{sec:intro}

% Importance of cities
% Importance of the spatial arrangement of cities
How we spatially arrange the building blocks that make up a city matters.
% Two sides of arranging stuff: form and function
These building blocks include the diffent
elements of the built and natural environments of which cities are composed,
but also the purpose they serve.
Understanding the former thus requires us to consider urban \textit{form}, while
grasping the latter invites us to examine its \textit{function}.
% FF encode history, technology and the values of the society that shapes
% them
Urban form and function are relevant for two main reasons. First, their fabric
encodes the socio-economic history, technology and values
of the society that has built them. 
%
The map of many European cities tells the story of the different historical
periods in which they were born, grew and, in some cases, contracted.
%
American cities which ``came of age'' in the second half of the XXth Century
would look very different had the autobomile technology not been the defining
development of the time.
%
And the stark contrasts between luxury developments and informal settlements
that can be observed across many cities of the Global South are a reflection
of the wide range of disparities and inequalities that those societies
display.
%
This encoding is also sticky. Although cities are constantly in flux, new
innovations and waves of change rarely start from scratch. More commonly, they
are added in a layered way. Over time, each phase, each change blends with the rest
of the urban tissue to give a city its uniquely distinct pattern that defines
it almost as a strand of DNA.
%
Studying the nature and distribution
of form and function in cities thus helps us better understand the societies
that, over time, have shaped them.
% but, once in place, it also has important implications
The second reason why urban form and function are important is that they are
not only a history book recording the past, but also play an important role in
defining the present and shaping the future.
%
Once in place, their features and structure have direct implications for a
wide range of outcomes, from  productivity and job access, to social inclusion
and mobility, deprivation, service provision, energy consumption or carbon
emissions, to name just a few.


% Gaps
%% fragmentation (academia/policy, disciplines)
%% need for detailed, consistent and scalable

% Opportunities
%% data
%% methodologies (morphometrics, GDS)
%% technology

% Preview
%% What: form and function
%% Benefits
%%% interdisciplinary, all-encompassing, not universal though!

% Academic relevance

% Relevance for policy

% Remainder structure of the paper




%--------------------------------------------------
% Bits to make sure to mention
%--------------------------------------------------

- need for detailed, consistent and scalable evidence (pick two of those)

- These needs relate to both developing world, where there is not data at all
  and most of the changes; but also to the developed world where life is being
  "recast" and cities continue to evolve (housing crisis, remote work, climate
  change targets, technology, etc.)

- Fragmented literature --> fragmented measurement --> fragmented evidence
- Fragmentation hinders our understanding

- SS are intellectually "all-ecompassing", get around fragmentation
- SS are fine-grained, consistent

- SS data-driven but theoretically informed; granular but scalable;and flexible enough to be adapted to a wide variety of applied contexts


\martin{Note: are we using capitalised Spatial Signatures or spatial signatures? It is inconsistent throughout the paper.}
