\section{Introduction}
\label{sec:intro}

% Importance of cities
% Importance of the spatial arrangement of cities
How we spatially arrange the building blocks that make up a city matters.
%
The map of many European cities tells the story of the different historical
periods in which they were born, grew and, in some cases, contracted.
%
American cities which ``came of age'' in the second half of the 20th Century
would look very different had the automobile not been the defining
technology of the time \citep{jacobs2016death}.
%
And the stark contrasts between luxury developments and informal settlements
that can be observed across many cities of the Global South are a reflection
of the wide range of disparities and inequalities that those societies
display \citep{alsayyad2003urban}.
%
This encoding of history, technology, and culture is also sticky. Once in place,
elements of the urban fabric change slowly over time. Although cities are constantly in flux, new
innovations and waves of change rarely start from scratch. More commonly, they
are added in a layered way. Over time, each phase, each change blends in with the rest
of the urban fabric to give a city its uniquely distinct pattern that defines
it almost as a strand of DNA.
% Two sides of arranging stuff: form and function
The building blocks of this process include the different
elements of the built and natural environments of which cities are composed,
but also the purpose they serve.
Understanding the former thus requires us to consider urban \textit{form}, while
grasping the latter invites us to examine its \textit{function}.
% FF encode history, technology and the values of the society that shapes
% them
Urban form and function are relevant for two main reasons. First, their fabric
encodes the socio-economic history, technology and values
of the society that has built them.
%
Studying the nature and distribution
of form and function in cities thus helps us better understand the societies
that, over time, have shaped them.
% but, once in place, it also has important implications
Second, urban form and function are
not only a history book recording the past, but also play an important role in
defining the present and shaping the future.
%
Once in place, their features and structure have direct implications for a
wide range of outcomes, from productivity and job access to social inclusion
and mobility, deprivation, service provision, energy consumption or carbon
emissions, to name just a few.
% 
In this context, the main contribution of the present paper is to
introduce the concept of spatial signatures as a characterisation of
space based on form and function designed to understand urban environments.
Our proposal advances the state-of-the-art by filling a series of gaps existing in the
current literature, including conceptual fragmentation and lack of detailed
evidence at scale.

% Gaps
The study of urban form and function is deeply fragmented
\citep{kropf2014ambiguity,brenner2015towards,gauthier2006mapping}.
%% fragmentation (academia/policy, disciplines)
Work with this focus is scattered across
different academic disciplines and policy-making scenes. This is not
necessarily a problem in itself since different backgrounds
provide a richer picture. And there is much to be gained from a plurality of
perspectives.
%
It does however mean that the evidence available presents different
interests as well as varying degrees of detail, consistency, and coverage.
% Academia
It is understandable that economists develop conceptualisations shaped around
economic theories (eg. \citealp{ahlfeldt2019}), while geographers do so paying attention
to spatial scales (eg. \citealp{boeing2018}),
and yet other disciplines bring different aspects to focus.
% Policy
Similarly, decision-makers interested in understanding aspects of urban form
and function tend to see it through the lenses provided by the vantage point
they occupy. Regional planners may try to obtain as much detail as
possible for a relatively small geographical area (eg. \citealp{bcnnight});
while supra-national organisations may prioritise scale and coverage
at the expense of detail (eg. \citealp{eea2016}).

%% need for detailed, consistent and scalable: currently, pick two
There is a clear need for detailed, consistent, and scalable evidence on urban
form and function.
% Why
Detailed and granular measurement that can be performed across large
geographical extents in a comparable way can unlock insights that get lost
when we can only observe certain regions of the picture developed for a large
extent does not have sufficient detail.
%
This is because many of the
theoretical underpinnings of urban form and function that reflect its history
and influence present and future outcomes tend to operate at fine scales but,
to be able to observe meaningful differences, we need to consider many and
different places.
%
For example, the characters that define Medieval city centres in Europe
quickly blur when the geographical unit considered is coarse. But, to be able to
examine how these characters relate to different levels of walkability, or
even of gas emissions, we require a large extent to set up meaningful
comparisons.

% What we currently have
Of detail, consistency and scale, the current research landscape described above
provides, at best, any two \citep{jochem2020,araldi2019,fleischmann2021methodological}.
%
There is an abundance of detailed studies that measure form and
function in an internally consistent way, but these are in their majority confined to
case studies with very limited geographical extents.
%
On the other end of the scale spectrum, recent years have seen the appearance
of work at a global scale that is internally consistent. However, their degree of detail
tends to be hindered by data limitations.
%
Finally, one could understand the multitude of detailed case studies in
conjunction as a growing body of evidence that is able to reach a sizeable
scale. But, in these case, the fragmentation discussed earlier often
translates in a lack of consistency that prevents meaningful comparisons.

% Opportunities
Recent advances in data, technology, and methodology are beginning to overcome
these limitations.
%% data
New forms of data such as open cadastres, consumer datasets derived from
modern business operations, or high resolution, public satellite imagery are
greatly improving the descriptions we can build of cities
\citep{arribas2014accidental, glaeser2018big, wei2020multiscale, fleischmann2021evolution}. Progressively, we are able to build denser and
more up to date representations of urban environments at a cheaper cost.
%% technology
Technological developments such as the dramatic increase of computational
power available to researchers, or improvements in computer algorithms and machine
learning are lowering the entry barrier to advances that only a few years ago
required a high degree of dedicated expertise to be able to benefit from.
%% methodologies (morphometrics, GDS)
Perhaps more importantly, recent methodological contributions such as
morphometrics \citep{dibble2016urban} or geographic data science
\citep{singleton2021geographic} are paving the way to blend these advances
with domain knowledge and urban theory, effectively enabling disciplines
concerned with the form and function of cities to benefit from such
developments.

% Preview
%% What: form and function
The spatial signatures are thus a delineation that divides geographical space
based on its appearance (form) and how it is used (function).
It is not a classification of space as much as a way of thinking around
classifying space based on form and function.
%%
The concept is data-driven but theoretically informed;
granular but scalable; and flexible enough to be adapted to a wide variety of
applied contexts, from data-rich to those with limited availability.
%
%% Provide a theoretically-rich, data-driven approach to understand cities
Spatial signatures embed theoretical ideas about how cities are
spatially arranged, how this configuration can be best conceptualised, and how
it is perceived by humans into a data-driven framework that connects them to the
vast amount of empirical information available representing the world.
%
%% Useful for many (``package urban form and function''), cross-disciplinary
These theoretical underpinnings are sourced from a variety of disciplines,
from architecture to environmental sciences, and thus are inherently
interdisciplinary.
%
%% Provide a shared vocabulary
The spatial signatures thus provide a shared vocabulary to bring together a
variety of scholars and policy makers for whom form and function in cities is
relevant, either as their main object of study or as an input for their own domains
of expertise.
%
%%
These characteristics make the spatial signatures an ideal candidate
for deployment on a wide range of data landscapes and geographical regions.
To demonstrate this flexibility, Section \ref{sec:app} presents five illustrations
that each apply the concept of spatial signatures to a different city, developing
internally consistent classifications for each of them.
%
It is important to emphasize we view these illustrations as a way to empirically
showcase our conceptual proposal rather than as the empirical, main contribution of
the paper.

% Relevance
In part because of their adaptability,
the spatial signatures hold important potential both for urban academics and
policy-makers.
%% Academic
From an academic point of view, they are relevant as a goal in themselves
that allows us to better measure and study the spatial configuration of the
building blocks that make up cities. But they also represent a platform on
which other disciplines can build on to embed form and function on a variety
of socio-environmental outcomes.
%% Policy
For policy-makers, the spatial signatures provide a framework for detailed
spatial understanding of the cities and territories their decisions affect.
They are useful both in the global
north, where cities are constantly recast and retrofitted, as well as the
global south, where most of the new urbanisation is currently taking place.
%
In summary, the spatial signatures allow us to move forward in realising
detailed, consistent, and scalable measurement of form and function in
cities.

% Remainder structure of the paper
The remaining of the paper is structured as follows. Section \ref{sec:lit}
reviews existing literature on urban form and function, highlighting current
gaps. Section \ref{sec:ss} represents the core of our contribution by detailing
our proposal of spatial signatures,
including how we define them, the spatial unit we develop to
measure them --the enclosed tessellation cell--, and the embedding of form and
function into such unit. In Section \ref{sec:app}, we illustrate the
flexibility of the spatial signatures by presenting five illustrations of the
spatial signatures to rather different global cities. And we conclude in Section
\ref{sec:conclusions} with some reflection about the value and potential of
our approach.



%--------------------------------------------------
% Bits to make sure to mention
%--------------------------------------------------



%+ These needs relate to both developing world, where there is not data at all
%  and most of the changes; but also to the developed world where life is being
%  "recast" and cities continue to evolve (housing crisis, remote work, climate
%  change targets, technology, etc.)


%+ SS are fine-grained, consistent
%+ SS data-driven but theoretically informed; granular but scalable;and flexible enough to be adapted to a wide variety of applied contexts
%x SS are intellectually "all-ecompassing", get around fragmentation
%+ need for detailed, consistent and scalable evidence (pick two of those)
%+ Fragmented literature --> fragmented measurement --> fragmented evidence
%+ Fragmentation hinders our understanding
