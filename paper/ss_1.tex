\section{Introduction}
\label{sec:intro}

% Importance of cities
% Importance of the spatial arrangement of cities
How we spatially arrange the building blocks that make up a city matters.
% Two sides of arranging stuff: form and function
These building blocks include the diffent
elements of the built and natural environments of which cities are composed,
but also the purpose they serve.
Understanding the former thus requires us to consider urban \textit{form}, while
grasping the latter invites us to examine its \textit{function}.
% FF encode history, technology and the values of the society that shapes
% them
Urban form and function are relevant for two main reasons. First, their fabric
encodes the socio-economic history, technology and values
of the society that has built them. 
%
The map of many European cities tells the story of the different historical
periods in which they were born, grew and, in some cases, contracted.
%
American cities which ``came of age'' in the second half of the XXth Century
would look very different had the autobomile not been the defining
technology of the time.
%
And the stark contrasts between luxury developments and informal settlements
that can be observed across many cities of the Global South are a reflection
of the wide range of disparities and inequalities that those societies
display.
%
This encoding is also sticky. Although cities are constantly in flux, new
innovations and waves of change rarely start from scratch. More commonly, they
are added in a layered way. Over time, each phase, each change blends with the rest
of the urban tissue to give a city its uniquely distinct pattern that defines
it almost as a strand of DNA.
%
Studying the nature and distribution
of form and function in cities thus helps us better understand the societies
that, over time, have shaped them.
% but, once in place, it also has important implications
The second reason why urban form and function are important is that they are
not only a history book recording the past, but also play an important role in
defining the present and shaping the future.
%
Once in place, their features and structure have direct implications for a
wide range of outcomes, from  productivity and job access, to social inclusion
and mobility, deprivation, service provision, energy consumption or carbon
emissions, to name just a few.


% Gaps
The study of urban form and function is deeply fragmented.
%% fragmentation (academia/policy, disciplines)
Work focusing on the form and function of cities is scattered across
different academic disciplines and policy-making scenes. This is not
necessarily a problem in itself, since different pespectives and backgrounds
provide a richer picture. And there is much to be gained from a plurality of
perspectives.
%
It does however mean that the evidence available presents different 
interests as well as varying degrees of detail, consistency, and coverage.
% Academia
It is understandable that economists develop conceptualisations shaped around
economic theories, while geographers do so paying attention to different
dimensions, and yet other disciplines bring different aspects to focus.
% Policy
Similarly, decision makers interested in understanding aspects of urban form
and function tend to see it through the lenses provided by the vantage point
they occuppy. Regional planners may try to obtain on as much detail as
possible for a relatively small geographical area; while supra-national
organisations may prioritise scale and coverage at the expense of detail.

%% need for detailed, consistent and scalable: currently, pick two
There is a clear need for detailed, consistent, and scalable evidence on urban
form and function.
% Why
Detailed and granular measurement that can be performed across large
geographical extents in a comparable way can unlock insights that get lost
when we can only observe certain regions or the picture developed for a large
extent does not have sufficient detail. 
%
This is because many of the
theoretical underpinnings of urban form and function that reflect its history
and influence present and future outcomes tend to operate at fine scales but,
to be able to observe meaningful differences, we need to consider many and
different places.
%
For example, the characters that define Medieval city centres in Europe
quickly blur when the geographical unit considered is coarse. But, to be able to
examine how these characters relate to different levels of walkability, or
even of gas emisions, we require a large extent to set up meaningful
comparisons.

% What we currently have
Of detail, consistency and scale, the current landscape described above
provides, at best, any pair.
%
There is an abundance of detailed studies that consistently measure form and
function, but these are in their majority confined to case studies with very
limited geographical extents.
%
On the other end of the scale spectrum, recents years have seen the appearance
of work at global scale that is consistent. However their degree of detail
tends to be hindered by data limitations.
%
Finally, one could understand the multitude of detailed case studies in
conjunction as a growing body of evidence that is able to reach a sizeable
scale. But, in these case, the fragmentation discussed earlier often prevents
meaningful comparisons.

% Opportunities
%% data
%% methodologies (morphometrics, GDS)
%% technology

% Preview
%% What: form and function
%% Benefits
%%% interdisciplinary, all-encompassing, not universal though!

% Academic relevance

% Relevance for policy

% Remainder structure of the paper




%--------------------------------------------------
% Bits to make sure to mention
%--------------------------------------------------

- need for detailed, consistent and scalable evidence (pick two of those)

- These needs relate to both developing world, where there is not data at all
  and most of the changes; but also to the developed world where life is being
  "recast" and cities continue to evolve (housing crisis, remote work, climate
  change targets, technology, etc.)

- Fragmented literature --> fragmented measurement --> fragmented evidence
- Fragmentation hinders our understanding

- SS are intellectually "all-ecompassing", get around fragmentation
- SS are fine-grained, consistent

- SS data-driven but theoretically informed; granular but scalable;and flexible enough to be adapted to a wide variety of applied contexts


\martin{Note: are we using capitalised Spatial Signatures or spatial signatures? It is inconsistent throughout the paper.}
