\section{Introduction}
\label{sec:intro}

% Importance of cities
% Importance of the spatial arrangement of cities
% Where function, in which form?

% Gaps
%% fragmentation (academia/policy, disciplines)
%% need for detailed, consistent and scalable
% Opportunities
%% data
%% methodologies (morphometrics, GDS)
%% technology

% Preview
%% What: form and function
%% Benefits
%%% interdisciplinary, all-encompassing, not universal though!

% Academic relevance

% Relevance for policy

% Remainder structure of the paper




%--------------------------------------------------
% Bits to make sure to mention
%--------------------------------------------------

- The spatial configuration of cities is related to productivity and job
  access, social inclusion and mobility, deprivation, service provision,
  energy consumption and carbon emissions, among others.

- need for detailed, consistent and scalable evidence (pick two of those)

- These needs relate to both developing world, where there is not data at all
  and most of the changes; but also to the developed world where life is being
  "recast" and cities continue to evolve (housing crisis, remote work, climate
  change targets, technology, etc.)

- Fragmented literature --> fragmented measurement --> fragmented evidence
- Fragmentation hinders our understanding

- SS are intellectually "all-ecompassing", get around fragmentation
- SS are fine-grained, consistent

- SS data-driven but theoretically informed; granular but scalable;and flexible enough to be adapted to a wide variety of applied contexts
