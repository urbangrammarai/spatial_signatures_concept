\section{Spatial Signatures}
\label{sec:ss}

% Benefits of blended FF
Despite the current sparsity of studies, we believe there are several benefits
in considering form and function in tandem when trying to understand urban
spaces.
% Gap for blended FF
%% FF are deeply interconnected, and some insights require the two
The two are deeply interconnected. This close
correlation implies that outcomes observed across form tend to hold true for
function, and viceversa. However, unique patterns emerge when particular types
of form and function come together. 
%% It is the combination of the two that encodes history, cultury, technology, etc.
In fact, we argue that it is only through the combination of form and function
that cities are able to encode and reflect sophisticated aspects of human
nature such as history, culture or technology.
%
In these cases, considering only one or the other hinders rather than enables,
as we risk missing important traits of the nature of a place.
%% From a pragmatic perspective, considering the two feeds more information, which leads to more robust representations
From a more empirical perspective, even when the two dimensions mostly
overlap, there is value in considering them jointly. Some aspects of form and
function are clear conceptually but challenging to measure. Broadening the
pool of indices that can be deployed ensures better accuracy when
characterising existing patterns on the ground.
%
In this section, we detail our proposal to understand urban form and function
through what we term “Spatial Signatures”. 


