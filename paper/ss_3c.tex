\subsection{Embedding form and function into spatial signatures}
\label{ssec:ss_ff}

%
This section covers how to develop spatial signatures once ET cells are
delineated for an area of study.
% - EC have to be described
ET cells take the role of the structural unit.
In themselves, they hold descriptive value in reflecting the configuration of
the urban environment. They also operate as a container, into which other
morphometric and functional characters can be embedded.
% - its character is used to grow Signatures
% - we aim to describe intrinsic character of each cell depending on itself and,
% importantly, its context
The next stage to develop spatial signatures is to build form and function
characters on top of ET cells. With this, we aim to describe both the
intrinsic traits of each cell depending on its own
geometry and nature, but to also include features of its immediate spatial
context.
% - some descriptors relate to form
To do this, we propose to collect a set of descriptors reflecting both form
and function to capture the essence and definition of spatial signatures.
%
This process will lead to a heterogeneous
mix of morphometric characters, capturing patterns of physical, built-up
environment; and functional characters, reflecting population, amenities,
land use classification or historical importance. It is to be noted that
every piece of information is considered within its spatial context.
%% - some descriptors relate to function
% - all, no matter the intinal granularity are linked to EC
% - direct descriptors - area etc.
% - proximity
It is not the role of this
section to provide a comprehensive list of all characters, morphometric and
functional, that would need to be derived. Such list will depend on the
specific context in which a spatial signatures classification is being
developed including, for example, data availability or nature of the
geographical area being considered.
%
However, it should always aim to reflect the nature of the form and function of each
place in as exhaustive a way as possible. Tables \ref{tab:bcn_form_excerpt} and
\ref{tab:bcn_fn_excerpt} contain an excerpt from tables \ref{tab:form_bcn} and
\ref{tab:fn_bcn} available in the supplementary material, to provide an illustration of
the nature of characters in question. The criteria of characters' selection build on the
principles explored by \cite{dibble2019origin} and \cite{fleischmann2021methodological},
following the rules proposed by \cite{sneath1973numerical}, that can be summarised as
\emph{include as many characters present in literature as is feasible, while minimising
potential collinearity and limiting redundancy of information}. That includes all
categories of form characters identified by \cite{fleischmann2020measuring} (dimension,
shape, spatial distribution, intensity, connectivity, diversity) and a wide range of
functional characters available for the specific case study (see the table
\ref{tab:bcn_fn_excerpt} and Supplementary material for details).

\small
\begin{longtable}{p{5cm}p{4cm}p{4cm}l}
\caption{The excerpt of form characters used in the Barcelona case study. The implementation details are available
in Jupyter notebooks available at <anonymised for peer-review>. The categorisation follows \cite{fleischmann2020measuring}.}
\label{tab:bcn_form_excerpt} \\
\toprule
                               index &                         element &                    context &     category \\
\midrule
\endfirsthead

\toprule
                               index &                         element &                    context &     category \\
\midrule
\endhead
\midrule
\multicolumn{4}{r}{{Continued on next page}} \\
\midrule
\endfoot

\bottomrule
\endlastfoot
\dots &                        \dots &                   \dots &    \dots \\
                                area &                        building &                   building &    dimension \\
                           perimeter &                        building &                   building &    dimension \\
                circular compactness &                        building &                   building &        shape \\
                          squareness &                        building &                   building &        shape \\
                   solar orientation &                        building &                   building & distribution \\
                    street alignment &                        building &                   building & distribution \\
                 coverage area ratio &               tessellation cell &          tessellation cell &    intensity \\
                            openness &                  street profile &             street segment & distribution \\
                              degree &                     street node &         neighbouring nodes & distribution \\
                  shared walls ratio &             adjacent buildings  &        adjacent buildings  & distribution \\
                                area &                       enclosure &                  enclosure &    dimension \\
                    local meshedness &                  street network &              nodes 5 steps & connectivity \\
          local closeness centrality &                  street network &              nodes 5 steps & connectivity \\
               perimeter wall length &             adjacent buildings  &           joined buildings &    dimension \\
               \dots &                        \dots &                   \dots &    \dots \\
\end{longtable}

\begin{longtable}{p{5cm}p{3cm}p{5cm}}
\caption{The excerpt of function characters and transfer methods used in the Barcelona case study. The implementation details are available
in Jupyter notebooks available at <anonymised for peer-review>.}
\label{tab:bcn_fn_excerpt} \\
\toprule
                                        character & input spatial unit &                                    transfer method \\
\midrule
\endfirsthead

\toprule
                                        character & input spatial unit &                                    transfer method \\
\midrule
\endhead
\midrule
\multicolumn{3}{r}{{Continued on next page}} \\
\midrule
\endfoot

\bottomrule
\endlastfoot
\dots &                        \dots &                   \dots  \\

                                        population &              block &                  Building-based Dasymetric mapping \\
    number of other items that are not premises &              block &                                 Dasymetric mapping \\
                                        land use &             parcel &                            Spatial join (centroid) \\
                            number of dwellings &           building &                                     Attribute join \\
                                            parks &             points & Accessibility  - distance to nearest / \# within 15min \\
                                    restaurants &              point & Accessibility  - distance to nearest / \# within 15min \\
                                            trees &             points &                               Spatial join (count) \\
                                            NDVI &          raster 1m &                                        Zonal stats \\
                                            \dots &                        \dots &                   \dots  \\

\end{longtable}
\normalsize

% - interpolation
% - joins
Collecting characters at the ET cell level is only half the task to develop
spatial signatures. Given the granularity and multi-dimensionality of the
information at this stage, we need to combine it in a way that retains its
core characteristics but is easier to parse through.
%
We propose a feasible aggregation of ET cells into
spatial signatures using unsupervised learning. Again, it is not the role of
this section to single out a technique, since many exist including K-Means,
gaussian mixture models, or self-organizing maps \citep{kohonen1990self}, to
name a few. We note there is no need to impose a spatial contiguity constraint
as spatially contiguous clusters of cells in the same signature will emerge
thanks to the inherent spatial autocorrelation of data derived from mutually
overlapping \textit{contexts}.
%
These continuous groups of cells grouped in the same cluster is what we call
instances of a spatial signature.
% - each EC is characterizes by form and function of its immediate surroundings,
% allowing a feasible aggregation of ECs to SS.
