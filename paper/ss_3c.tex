\subsection{Embedding form and function into spatial signatures}
\label{ssec:ss_ff}

%
This section covers the development of spatial signatures from a set of ET cells.
% - EC have to be described
ET cells take the role of the structural unit.
In themselves, they hold descriptive value in reflecting the configuration of
the urban environment. They also operate as a container, into which other
morphometric and functional characters can be embedded.
% - its character is used to grow Signatures
% - we aim to describe intrinsic character of each cell depending on itself and,
% importantly, its context
To "fill" these containers with more information,
we propose to collect a set of descriptors reflecting both form
and function so that, together, they capture an intertwined representation
of both dimensions.
%
In practice, this process will lead to a heterogeneous
mix of morphometric characters, capturing patterns of physical, built-up
environment; and functional characters, reflecting economic activity, amenities,
land use classification or historical importance.

\small
\begin{longtable}{p{5cm}p{4cm}p{4cm}l}
\caption{Excerpt of form characters used in the Barcelona case study. Implementation details are provided
in Jupyter notebooks available at \texttt{<anonymised for peer-review>}. The categorisation follows \cite{fleischmann2020measuring}.}
\label{tab:bcn_form_excerpt} \\
\toprule
                               index &                         element &                    context &     category \\
\midrule
\endfirsthead

\toprule
                               index &                         element &                    context &     category \\
\midrule
\endhead
\midrule
\multicolumn{4}{r}{{Continued on next page}} \\
\midrule
\endfoot

\bottomrule
\endlastfoot
\dots &                        \dots &                   \dots &    \dots \\
                                area &                        building &                   building &    dimension \\
                           perimeter &                        building &                   building &    dimension \\
                circular compactness &                        building &                   building &        shape \\
                          squareness &                        building &                   building &        shape \\
                   solar orientation &                        building &                   building & distribution \\
                    street alignment &                        building &                   building & distribution \\
                 coverage area ratio &               tessellation cell &          tessellation cell &    intensity \\
                            openness &                  street profile &             street segment & distribution \\
                              degree &                     street node &         neighbouring nodes & distribution \\
                  shared walls ratio &             adjacent buildings  &        adjacent buildings  & distribution \\
                                area &                       enclosure &                  enclosure &    dimension \\
                    local meshedness &                  street network &              nodes 5 steps & connectivity \\
          local closeness centrality &                  street network &              nodes 5 steps & connectivity \\
               perimeter wall length &             adjacent buildings  &           joined buildings &    dimension \\
               \dots &                        \dots &                   \dots &    \dots \\
\end{longtable}


 
Which exact characters to compile for a particular implementation of spatial signatures
will depend on the availability of data in that context.
Since this section outlines the process conceptually, we do not consider including any specific
list as useful as providing broad guidance on the kind of characters that should 
be aimed for when designing an application of the spatial signatures.
Any selection in this respect should aspire to reflect the nature of form and function in the
area of interest in as exhaustive a way as possible.
% Form/function example
As an example,
Table \ref{tab:bcn_form_excerpt} (\ref{tab:bcn_fn_excerpt}) contains an excerpt of Table
\ref{tab:form_bcn} (\ref{tab:fn_bcn}) in the supplementary
material, which captures all of the form (function) characters we use in the Barcelona illustration of
Section \ref{sec:app}.
%
We recommend building on the principles explored by \cite{dibble2019origin} and
\cite{fleischmann2021methodological}, and following the rules originally proposed by
\cite{sneath1973numerical}. These can be broadly summarised as
\emph{include as many characters present in literature as is feasible, while minimising
potential collinearity and limiting redundancy of information}. This guidance includes all
categories of form characters identified by \cite{fleischmann2020measuring} (ie. dimension,
shape, spatial distribution, intensity, connectivity, diversity) and as wide as possible of a range
for functional characters available for a given case study, including land cover/use,
employment/economic activity, and amenities. 
%
While the optimal ratio of form to function characters is to be aimed at balance, this may not
always be possible. Form characters can be derived from a small set of data sources (eg.
street networks and building footprints) while describing function relies on a larger set
of data. We do not see this as a limitation of the spatial signatures as much as one
of data availability. If anything, the joint approach of form and function encouraged
by our proposal ameliorates the problem given the interrelations between form and function
described above and that function has been found to be implicitly present in the description of form
(\citealp{caniggia2001architectural} in \citealp{kropf2009aspects}).

\begin{longtable}{p{5cm}p{3cm}p{5cm}}
\caption{Excerpt of function characters and transfer methods used in the Barcelona case study.
Implementation details are provided
in Jupyter notebooks available at \texttt{<anonymised for peer-review>}.}
\label{tab:bcn_fn_excerpt} \\
\toprule
                                        character & input spatial unit &                                    transfer method \\
\midrule
\endfirsthead

\toprule
                                        character & input spatial unit &                                    transfer method \\
\midrule
\endhead
\midrule
\multicolumn{3}{r}{{Continued on next page}} \\
\midrule
\endfoot

\bottomrule
\endlastfoot
\dots &                        \dots &                   \dots  \\

                                        population &              block &                  Building-based Dasymetric mapping \\
    number of other items that are not premises &              block &                                 Dasymetric mapping \\
                                        land use &             parcel &                            Spatial join (centroid) \\
                            number of dwellings &           building &                                     Attribute join \\
                                            parks &             points & Accessibility  - distance to nearest / \# within 15min \\
                                    restaurants &              point & Accessibility  - distance to nearest / \# within 15min \\
                                            trees &             points &                               Spatial join (count) \\
                                            NDVI &          raster 1m &                                        Zonal stats \\
                                            \dots &                        \dots &                   \dots  \\

\end{longtable}
\normalsize

                %%% Context %%%
\dani{Martin add paragraph on how we use context and why here}
In this step our proposal is to describe both the
intrinsic traits of each cell, but to also incorporate features from the immediate
spatial context.
%
It is to be noted that
every piece of information is considered within its spatial context.
%----------------------------------------------------------------------------------

% - interpolation
% - joins
Collecting characters at the ET cell level is only half the task to develop
spatial signatures. Given the granularity and multi-dimensionality of the
information at this stage, we need to combine it in a way that retains its
core characteristics but is easier to parse through.
%
We propose a feasible aggregation of ET cells into
spatial signatures using unsupervised learning. Again, it is not the role of
this section to single out a technique, since many exist including K-Means,
gaussian mixture models, or self-organizing maps \citep{kohonen1990self}, to
name a few. We note there is no need to impose a spatial contiguity constraint
as spatially contiguous clusters of cells in the same signature will emerge
thanks to the inherent spatial autocorrelation of data derived from mutually
overlapping \textit{contexts}.
%
These continuous groups of cells grouped in the same cluster is what we call
instances of a spatial signature.
% - each EC is characterizes by form and function of its immediate surroundings,
% allowing a feasible aggregation of ECs to SS.
