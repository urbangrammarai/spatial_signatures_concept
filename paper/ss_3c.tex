\subsection{Embedding form and function into Spatial Signatures}
\label{ssec:ss_ff}

% - EC have to be described
Enclosed tessellation cells take the role of a structural unit, which itself has a
descriptive value reflecting configuration of the built environment, and a
container, into which other morphometric and functional characters can be
embedded.
% - its character is used to grow Signatures
% - we aim to describe intrinsic character of each cell depending on itself and,
% importantly, its context
We aim to describe the intrinsic character of each cell depending on its own
geometry and nature, and, importantly, its immediate spatial context.
% - some descriptors relate to form
The set of descriptors has to reflect both form and function to reflect the
essence and definition of spatial signatures, leading to a heterogeneous
selection of morphometric characters, capturing patterns of physical, built-up
environment, and functional characters, reflecting demographics, proximity to
points of interest, land use classification or historical importance.
% - some descriptors relate to function
% - all, no matter the intinal granularity are linked to EC
% - direct descriptors - area etc.
% - proximity
The specific composition of descriptors is not set and should always react to
local conditions in terms of both data availability (e.g. to capture morphometric characters
requiring high granularity or known building height may not be possible) and
the specificity of each place (e.g. you would not capture proximity to theatres in
a rural case study where are none but you would in metropolitan areas). However,
it should always aim to reflect the nature of the form and function of each
place in an exhaustive way.
% - interpolation
% - joins
The inherent spatial autocorrelation of data derived from mutually overlapping
\textit{contexts} of each EC then allows a feasible aggregation of ECs into
spatial signatures using cluster analysis, as K-Means, or Kohonen Self-organized
Map, without explicit spatial contiguity constraint.
% - each EC is characterizes by form and function of its immediate surroundings,
% allowing a feasible aggregation of ECs to SS.