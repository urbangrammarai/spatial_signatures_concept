\section{Illustration}
\label{sec:app}

% when talking about BCN mention
% https://www.researchgate.net/publication/254457523_Urban_form_and_compactness_of_morphological_homogeneous_districts_in_Barcelona_towards_an_automatic_classification_of_similar_built-up_structures_in_the_city#

% all technical details in to an appendix tables from excel should go to
% notebooks

\subsection{Method}
%%%%%%%%%%%% Structure 4.1 (squeeze it in a page) introduction of cases - 1 per
% continent; geographical variation, take different cities, cultures, historical
% moments

% spatial signature is a conceptual framework which can materialise in different
% ways
The classification of built form into spatial signatures is a conceptual
framework for characterisation of the built environment and as such, can
materialise in different ways based on the particular implementation of a
description of both form and function as well as the clustering method.
% here we present it applied to five cases from around the world
Here we present the concept applied to five case studies, reflecting different
environments to be classified and heterogeneous input data requiring the
adaptation of the classification to individual situations.
% the sample comes with geographical variation - 1, 2, 3, 4, 5
The sample offers a geographical variation covering Europe (Barcelona, Spain),
North America (Houston, TX, United States), South America (Medellin, Colombia),
Africa (Dar es Salaam, Tanzania) and South-east Asia (Singapore),
% that brings different culture, development, planning paradigm, historical and
% social contexts
coupled with cultural diversity, different planning paradigms involved in
shaping the respective environments as well as varied historical and social
contexts in which the selected cities were built.
% at the same time, it brings different input data, varied in both quality and
% richness
At the same time, the selection brings a variety of input data covering both
extremes in terms of quality (e.g., official mapping in Barcelona vs remote
sensing in Houston), the richness of information on functional aspects of places
(e.g., detailed data on the municipal level in Medellin vs global gridded
datasets in Dar es Salaam) and scale (82,375 units in Barcelona vs 2,043,581 units in
Houston).
% we present this variety to illustrate the application and flexibility of the
% concept
We present this variety to illustrate the flexibility of spatial signatures to
accommodate varied inputs and adapt to a local specificity, while retaining the
merit of the concept.

%% method - top level outline of the method. Data, Form + convolution, Function,
% Clustering One para on how to build the data - F+F+Convolution, how it varies
% across examples (link to App.)

%%% shall we add conceptual diagram of the method?

% 1. data retrieval from open sources
The delineation of spatial signatures starts with the input data reflecting form
and function of each place. The form is represented by building footprints and
physical barriers denoting streets, railways, and water bodies.
% 2. geographies - enclosures, enclosed tessellation
Using barriers, we first identify the geometry of enclosures to determine the
external boundaries of consequently generated enclosed tessellation.
% 3. characterization of form - primary + convolutions
Such information is rich enough for a comprehensive morphometric analysis
composed of primary measurable characters, capturing individual aspects of form,
and contextualisation, following the model proposed by Fleischmann et al. (2021)
(ADD REF).
We measure first, second, and third quartile of each
character's distribution within 10th order of contiguity on enclosed
tessellation weighted by inverse distance for each enclosed cell. Such adaptive
topological aggregation reflects the notion of context around each cell and
describes each character's tendencies within it.
% 4. characterisation of function
Function is captured as a heterogenous set of datasets relfecting aspects from
population to location of points of interest. All aspects are linked to enclosed
cells using the most appropriate method for each data input (e.g., areal
interpolation or network accessibility).\footnote{See the technical appendix for
details on the implementation.}

% Second para on how we generate signatures (clustering); once we have those we
% dissolve.

% 5. cluster analysis input (all linked to cell)
Spatial signatures are then identified using cluster analysis combined with the
notion of contiguity, where each contiguous portion of land belonging to a
single cluster is seen as a single signature.
% 6. clustergram
The combined data reflecting both form and function are therefore standardised
and clustered using K-Means clustering. Since the number of classes is not known
a priori, we use clustergram (schonlau) to understand clustering behaviour
within different options and select the optimal number according to its
structure.
% 7. clustering
The final clustering is run with 1000 initialisations to ensure the stability of
the results.
% 8. dissolution
The geometry of each spatial signature is then derived as dissolution of a
contiguous patch of enclosed tessellation within the same cluster.


\subsection{Results}
% 4.2 (a page and a half) Results; tell some stories to get reader on board

%%% Para1 how to read the map in an applied way
% geometries reflect boundaries of spatial signatures
Figures XXX-YYY illustrate the resulting spatial signatures in the respective case
studies.\footnote{For intermediate steps (e.g. clustergram) please refer to the
technical appendix.} The geometries reflect the spatial extent of individual
signatures derived from the enclosed tessellation
% colours reflect a type of a signature - two areas within the same type share
% the characteristics; similarity of colours has no meaning
with colour coding reflecting the type of a signature, i.e. the initial cluster. Two
areas within the same type are expected to share the characteristics of built
environment, being more similar (not necessarily the same) to each other than to
the rest of the classes. Note that the similarity of different colours does not
encode similarity of signatures. Note that due to varied extent of case studies,
maps are not printed at the same scale.

%%% Para2 things which are shared/consistent
% 9-19 clusters, not dependent on the scale but reflecting the heterogeneity of
% a place
The granularity of classification ranges from 9 (Houston) to 19 (Medellin)
signature types per case study. However, the actual number is not dependent on
the size of each city but rather on each place's actual heterogeneity, best
illustrated on the comparison of Houston and Barcelona, the largest (2 million
cells) and the smallest (80 thousand cells) case. Houston, representing north
American sprawling urban fabric shows a considerably smaller diversity of
spatial patterns (9 types of spatial signatures) than Barcelona (16 types),
reflecting the richness of their respective historical developments.
% core clusters and outlier clusters - the abundance varies a lot
The distribution of cluster sizes follows the same pattern of unequal abundance
across all cases. The most extensive types contain between 15 and 28% of all
observations, and the abundance is gradually decreasing towards a small number
of outlier clusters containing less than a per cent of all observations within
each sample.
% transition from core to countryside - singapore is exception, its island
% nature restricts such behaviour
All the cases clearly defined both extremes on the urbanisation axis, with
delineated central districts on the one hand and non-urban countryside
signatures on the other. The transition between the two tends to follow the
gradual pattern of signatures each less urban than the previous. The only
exception where this tendency is not so profound (but still present) is
Singapore, which geographical extent limited to the defined area of the main
island does not allow the full transition.


%%% Para3 interesting stories from cases (BCN village centres, DeS slums)
% BCN reflects its historical origin - pre-industrial core and village centes,
% later filled by eixample, which has its homogenous core and areas blending
% into the existing fabric
Barcelona is known for its industrial grid, which is captured as a unique
signature. However, the Cerda’s grid is historically an infill between the
city's medieval core and smaller existing settlements around. Both core and
former independent villages are reflected in the typology of signatures, which
reflect the historical origin of distinct places. The transition between the
two, the historical organic fabrics and rigid Eixample is reflected as another
signature, stitching together different patterns into a coherent city.
% Medellin's change of the pattern as we go up the hill from the central valley
% - central plateu allows more rigid planning reflected in several classes,
%   hillsides are becoming more vernacular
Spatial distribution of signatures in Medellin tells the story of its intricate
topography, even though the input data do not contain any information on
altitude. The city lies in the valley surrounded by steep slopes. While the
central parts lie on the relatively flat floor allowing paradigmatic planning
and rigidness of the built environment, hillsides are becoming more vernacular
leading to a sharp urban edge where the topography does not allow further
development.

% Dar es Salaam signatures capture different levels of formality, planned,
% semi-formal to informal
Signatures in Dar es Salaam reflect the changes in the formality of development,
with formal areas distributed in the central parts of the city in the vicinity
of a coastline. The transition between different degrees of formality is not
always gradual as the most informal parts of the city are infills of the space
not occupied by more planned neighbourhoods.

% houston and its homogenous sprawl, changing its nature of sprawlness along the
% time and growth, with major roads forming spines of non-residential development
The character of spatial signatures in Houston follows two primary principles.
One type forms the spine of activity spreading from the city centre radially to
the suburbs. The other, filling the areas in between the former, is a story of
the deterioration of compact, walkable urban block into convoluted dendritic
street network patterns of modern suburbs. The change in these predominantly
residential signatures is gradual and reflect the waves of development of the
city as it was growing over the years.
% NOTE: Houston signatures are not able to distinguish CBD from the rest of the
% "spine", shall we dicuss that at some point?

% Singapore singature tell the story of its growth. we can follow individual
% clusters and link their origin to different time periods
A similar situation is in Singapore, where different types of signatures can be
linked to the period of the origin of the development of each specific
neighbourhood. Contrary to previous cases, the development and, consequently,
spatial signatures followed radial manner, not entirely contiguous, with major
infills built in the last 50 years.
