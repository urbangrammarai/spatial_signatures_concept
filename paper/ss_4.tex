\section{Illustration}
\label{sec:app}

% when talking about BCN mention
% https://www.researchgate.net/publication/254457523_Urban_form_and_compactness_of_morphological_homogeneous_districts_in_Barcelona_towards_an_automatic_classification_of_similar_built-up_structures_in_the_city#

% all technical details in to an appendix tables from excel should go to
% notebooks

%%%%%%%%%%%% Structure 4.1 (squeeze it in a page) introduction of cases - 1 per
% continent; geographical variation, take different cities, cultures, historical
% moments

% spatial signature is a conceptual framework which can materialize in different
% ways
The classification of built form into spatial signatures is a conceptual
framework for characterisation of the built environment and as such, can
materialise in different ways based on the particular implementation of a
description of both form and function as well as the clustering method.
% here we present it applied to five cases from around the world
Here we present the concept applied to five case studies, reflecting different
environments to be classified and heterogeneous input data requiring the
adaptation of the classification to individual situations.
% the sample comes with geographical variation - 1, 2, 3, 4, 5
The sample offers a geographical variation covering Europe (Barcelona, Spain),
North America (Houston, TX, United States), South America (Medellin, Colombia),
Africa (Dar es Salaam, Tanzania) and South-east Asia (Singapore),
% that brings different culture, development, planning paradigm, historical and
% social contexts
coupled with cultural diversity, different planning paradigms involved in
shaping the respective environments as well as varied historical and social
contexts in which the selected cities were built.
% at the same time, it brings different input data, varied in both quality and
% richness
At the same time, the selection brings a variety of input data covering both
extremes in terms of quality (e.g., official mapping in Barcelona vs remote
sensing in Houston), the richness of information on functional aspects of places
(e.g., detailed data on the municipal level in Medellin vs global gridded
datasets in Dar es Salaam) and scale (82,375 units in Barcelona vs 2,043,581 units in
Houston).
% we present this variety to illustrate the application and flexibility of the
% concept
We present this variety to illustrate the flexibility of spatial signatures to
accommodate varied inputs and adapt to a local specificity, while retaining the
merit of the concept.

%% method - top level outline of the method. Data, Form + convolution, Function,
% Clustering One para on how to build the data - F+F+Convolution, how it varies
% across examples (link to App.)

%%% shall we add conceptual diagram of the method?

% 1. data retrieval from open sources
The delineation of spatial signatures starts with the input data reflecting form
and function of each place. The form is represented by building footprints and
physical barriers denoting streets, railways, and water bodies.
% 2. geographies - enclosures, enclosed tessellation
Using barriers, we first identify the geometry of enclosures to determine the
external boundaries of consequently generated enclosed tessellation.
% 3. characterization of form - primary + convolutions
Such an information is rich enough for a comprehensive morphometric analysis
composed of primary measurable characters, capturing individual aspects of form,
and contextualisation. We measure first, second, and third quartile of each
character's distribution within 10th order of contiguity on enclosed
tessellation weighted by inverse distance for each enclosed cell. Such adaptive
topological aggregation reflects the notion of context around each cell and
describes each character's tendencies within it.
% 4. characterization of function
Function is captured as a heterogenous set of datasets relfecting aspects from
population to location of points of interest. All aspects are linked to enclosed
cells using the most appropriate method for each data input (e.g., areal
interpolation or network accessibility).\footnote{See the technical appendix for
details on the implementation.}

% Second para on how we generate signatures (clustering); once we have those we
% dissolve.

% 5. cluster analysis input (all linked to cell)
Spatial signatures are then identified using cluster analysis combined with the
notion of contiguity, where each contiguous portion of land belonging to a
single cluster is seen as a single signature.
% 6. clustergram
The combined data reflecting both form and function are therefore standardized
and clustered using K-Means clustering. Since the number of classes is not known
a priori, we use clustergram (schonlau) to understand clustering behaviour
within different options and select the optimal number according to its
structure.
% 7. clustering
The final clustering is run with 1000 initializations to ensure the stability of
the results.
% 8. dissolution
The geometry of each spatial signature is then derived as dissolution of a
contiguous patch of enclosed tessellation within the same cluster.

% 4.2 (a page and a half) Results; tell some stories to get reader on board

% how to read the map in an applied way

% things which are shared/consistent

% interesting stories from cases (BCN village centers, DeS slums)
